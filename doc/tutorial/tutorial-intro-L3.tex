

\question{Connect to the public server (\url{https://app.vidjil.org}), either with your account
or the demo account (\texttt{demo@vidjil.org} / \texttt{demo}),
select the \textit{Demo LIL-L3 (tutorial)} patient.
If you don't see it, search for \com{\#Demo} in the top-left search box.
Then click on the bottom right link, \com{see results: multi-inc-xxx}.
Do not open the \textit{Demo LIL-L3 (analyzed)} patient: this one contains the complete
analysis.
The Vidjil web application opens.}

This patient (patient 063 from \href{http://dx.doi.org/10.1016/j.leukres.2016.11com.009}{Lille study on the feasibility of MRD using HTS}) suffering T-ALL has one diagnosis sample,
with dominant clones both in IGH and TRG,
and four follow-up samples, including a relapse.

\question{In the \com{settings} menu, try the various options for \com{sample key}.
  The five samples can be labeled by their name, their date of sampling or by the number of days after the first sample.
}

In the following sections, we focus on the diagnosis sample.
% and explore how to assess the quality of the data
% how to view and filter clones and
The section~\ref{sec:tracking} will deal with the comparison of several samples. 

