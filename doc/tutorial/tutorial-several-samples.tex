

\section{Tracking clones on several samples}

\label{sec:tracking}
\begin{verbatim}


  // ######################################
  // ### Tracking clones on several samples
  // ######################################

\end{verbatim}

Load now some data with several samples, such as again the \textit{Demo LIL-L3} dataset.
The \textit{time graph} shows the evolution of the top clones of each sample into all the samples.
Bear in mind that to ensure readability at most 10 curves are displayed in this graph by time point.
\marginpar{When loading data with only one sample, the time graph is replaced by a second bar/grid plot.}

\question{Pass the mouse over the bubbles in the grid or over the lines in the time graph.
  Click on some clone. What happens ?}
\begin{verbatim}

  ////////////////////
  // test_C_samples_01
    cy.selectClone(0)
    cy.get("#visu_circle1").should("exist").should("be.visible")
    cy.get("#visu_circle1").trigger('mouseover')
    cy.get("#visu_circle1").should("have.class", "circle_focus")
    cy.get('#\\31 ').should("have.class", "list_focus")
    cy.get('#polyline1').should("have.class", "graph_focus")
    cy.get('.focus').should("have.text", "interesting clone (IGHV3-9*01 7/CCCGGA/17 IGHJ6*02)")


\end{verbatim}

\question{Click on the label of the time graph to select another sample.
  What happens to the number of analyzed reads ? to the size of the top clones
  ?}
\begin{verbatim}
  ////////////////////
  // test_C_samples_02
    cy.selectClone('1')

    // control init state
    cy.get('#info_sample_name').should("have.text", "LIL-L3-0")
    cy.getInfoAnalysedReads().should("have.text", "1 967 338 (82.31%)")
    cy.getInfoSelectedLocus().should("have.text", "1 965 646 (82.24%)")
    cy.getCloneSize("1").should('have.text', "9.665%")

    // change time point
    cy.get("#time1").click()
    cy.get('#info_sample_name').should("have.text", "LIL-L3-1")
    cy.getInfoAnalysedReads().should("have.text", "1 892 104 (77.19%)")
    cy.getInfoSelectedLocus().should("have.text", "1 889 135 (77.07%)")
    cy.getCloneSize("1").should('have.text', "−")


\end{verbatim}

When switching the time point, the views dynamically update which allows to
easily track the changes along time. Also note that the number of analyzed
reads differ from the previous point. We can again analyse the reason why some
reads were unsegmented.

\question{Hover labels of the time graph. What are you seeing ? }

\bigskip

We will look now at how the V gene distribution evolves along the time.
\question{In the grid, select the preset \com{V distribution}. Then click
  on the \com{play} icon in the upper left part (below the \textit{i} icon).}
\begin{verbatim}
  ///////////////////////////
  // test_C_samples_03_play_
    cy.get("#time0").click()
    cy.get('#info_sample_name').should("have.text", "LIL-L3-0") //Correct name of sample before play lunched

    // launch "play" function
    cy.get('.icon-play').click()
    cy.get('.icon-pause').should("exist").should("be.visible")
    cy.get('#info_sample_name').should("have.text", "LIL-L3-1", "Correct name after play, step 1")
    cy.wait(2000)
    cy.get('#info_sample_name').should("have.text", "LIL-L3-2", "Correct name after play, step 2")
    cy.wait(2000)
    cy.get('#info_sample_name').should("have.text", "LIL-L3-3", "Correct name after play, step 3")
    cy.wait(2000)
    cy.get('#info_sample_name').should("have.text", "LIL-L3-4", "Correct name after play, step 4")
    cy.wait(2000)
    cy.get('#info_sample_name').should("have.text", "LIL-L3-0", "Correct name after play, step 5; return to first sample")


\end{verbatim}

By doing so you can look at how the V distribution changes along the time.
Of course you can also change the data displayed in the grid to look at
the evolution of another information.

\bigskip

We remind that by default at most 50 clones are displayed
% verifier ça.
on the time graph. However the remaining of the application usually displays
the 50 \textit{most abundant clones} at each sample (which can account to hundreds of
clones, when having several samples).

\question{Select a sample, order the list by size, and pass the mouse through the list
  of top 50 clones. What happens in the graph when hovering clones that are not in the top 50 ?}
\begin{verbatim}
  ////////////////////////////////////
  // test_C_samples_04_lock_order_list
    cy.get('#div_sortLock').should("have.class", "icon-lock-1 list_lock_on") // lock start in good state (locked)"
    cy.changeSortList("size")

    cy.get('#div_sortLock').click()
    cy.get('#div_sortLock').should("have.class", "icon-lock-open list_lock_off") // lock is in the good state after click (open)"

    cy.get('#list_clones > :nth-child(1)').should("have.attr", "id", "453") // correct id of the first element (smaller TRG clones)

    cy.get("#time4").click()
    cy.get('#list_clones > :nth-child(1)').should("have.attr", "id", "0") // correct id of the first element after changing timepoint


\end{verbatim}

\bigskip

If you have many samples, you may wish to reorder the samples.

\question{Drag the label of one sample to reorder the samples.}
\begin{verbatim}
  ////////////////////
  // test_C_samples_05
    // TODO


\end{verbatim}

It is possible to hide some samples to have a better view of your analysis. This can be very usefull to hide for example samples without a clone.
Another case of usage is to load a run analysis with a lot of sample, and to restrict the list of active samples to retain only one that share a particular clone.

Open the menu to the upper right corner of graph. you can see that their are 3 buttons and 5 entries in the list corresponding to the list of samples.
See the effet of the button here
The name dispayed in the list is set accordingly with the settings choice for \textit{samples key}.
\question{Hover a sample name in the list. See what happend.}
\begin{verbatim}
  ////////////////////
  // test_C_samples_06
    // TODO


\end{verbatim}

\question{It is possible to switch the current sample showed by clicking on corresponding line LIL-L3-3. What happened ?}
\begin{verbatim}
  ///////////////////////////////////////////////////
  // test_C_samples_07_switch_current_sample_by_click
    cy.get("#time0").click()
    cy.get('#info_sample_name').should("have.text", "LIL-L3-0", "Correct name of sample at init")
    cy.get("#visu2_listElem_3").click({force: true})
    cy.get('#info_sample_name').should("have.text", "LIL-L3-3", "Correct name of sample after click on the corresponding row")
    cy.get("#visu2_listElem_3").dblclick({force: true})
    cy.get('#info_sample_name').should("have.text", "LIL-L3-0", "Correct name of sample after sample switch (double_click)")

    cy.get("#visu2_listElem_check_3").click({force: true})
    cy.get('#info_sample_name').should("have.text", "LIL-L3-3", "Correct name of sample after checkbox switch (activate LIL-L3-3)")


\end{verbatim}
It is also possible to switch on/off the status of a sample by clicking on the checkbox, double clicking on the line, or double clicking on the label in the timeline graph.

\bigskip

You may also want to compare two samples, either to check a replicate, to check for possible contaminations, or to
compare different research or medical situations.

\question{In the \com{color by} menu, choose \com{by size}. Select a different
  sample. What happens ? Are there some clones with a significant different concentration in both samples ?
Revert the color by choosing \com{by tag}.}
\begin{verbatim}
  ///////////////////////////////////
  // test_C_samples_08_compare_sample
    cy.changeColorby("Size")
    cy.get('.gradient').should('exist', "color gradient is present")
    cy.getCloneInList(0).should('have.css', 'color', 'rgb(174, 184, 0)')

    cy.get("#time4").click()
    cy.getCloneInList(0).should('have.css', 'color', 'rgb(246, 73, 0)')


\end{verbatim}

Another option is to directly plot a log-log curve comparing two samples.

\question{In the \com{plot} menu, choose the preset \com{compare two samples}. Click
  successively on two labels in the time graph to select the samples to be compared.
  Are there again some clones with a significant different concentration in both samples ?}
\begin{verbatim}
  ///////////////////////////////////
  // test_C_samples_09_compare_sample
    cy.get("#time0").click() // set previous sample
    cy.get("#time4").click() // set current sample
    cy.get("body").type(9)

    cy.get('#id_legend_x').should("have.text", "Size", "Correct legend for axe X")
    cy.get('#id_legend_y').should("have.text", "Size (other)", "Correct legend for axe Y")

    cy.get('#id_label_x_0').should("exist").should("be.visible", "correct legend for axe X at pos 0")
    cy.get('#id_label_x_1').should("exist").should("be.visible", "correct legend for axe X at pos 8 (biggest)")
    cy.get('#id_label_y_0').should("exist").should("be.visible", "correct legend for axe Y at pos 0")
    cy.get('#id_label_y_0\\.0000001').should("exist").should("be.visible", "correct legend for axe Y at pos 7 (smallest)")

/**
\end{verbatim}

