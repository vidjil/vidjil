\section{Viewing and filtering clonotypes}
\label{sec:viewing-clonotypes}

First-time users are advised to have a look at \href{https://www.vidjil.org/doc/user/#the-elements-of-the-vidjil-web-application}{\tt the-elements-of-the-vidjil-web-application} that describes the main elements of Vidjil.
% Le mettre en amont ? 

\subsection{Looking at a clonotype}
\begin{verbatim}


  // #################################
  // ### Viewing and filtering clones
  // #################################

\end{verbatim}


Each RepSeq algorithm has its own definition of what a clonotype is (or, more precisely
a clonotype), and on how to output its sequence and how to assign a V(D)J designation.

In this sample, the most abundant clonotype
is \texttt{IGHV3-9 7/CCCGGA/17 J6*02}.

\question{Select this clonotype, either by clicking on the list or on the grid.
  How many reads do this clonotype represent? (see again the bottom part to the right)}

\reponse{The bottom panel display information about currently selected clonotypes \fl 189 991 reads (9.665\,\%)}

\begin{verbatim}
  /////////////////////////
  cy.log("// test_B_clones_01_major")
  cy.selectClone('1')
  cy.get('.focus').should('have.text', "IGHV3-9*01 7/CCCGGA/17 IGHJ6*02")
  cy.get('.stats_content').should('have.text', '1 clonotype, 189 991 reads (9.665%) ')

\end{verbatim}


There are several options to display the V(D)J designation.

\question{In the  \com{settings} menu, under \com{N regions in clonotype names} select \com{length} to show N zones by their length. Revert to the
  default \com{sequence (when short)} setting to show the full N on short sequences.}

\begin{verbatim}
  ////////////////////////////////////
  cy.log("// test_B_clones_02_setting_n_length")
  cy.get('#menuCloneNot_nucleotide_number')
    .click({force: true})
  cy.get('#listElem_1 > .nameBox')
    .should('have.text', "IGHV3-9 7/6/17 J6*02")
  cy.get('#listElem_30 > .nameBox')
    .should('have.text', "IGHV3-13*05 1/55/16 J6*02")

  cy.get('#menuCloneNot_short_sequence')
    .click({force: true})
  cy.get('#listElem_1 > .nameBox')
    .should('have.text', "IGHV3-9 7/CCCGGA/17 J6*02")
  cy.get('#listElem_30 > .nameBox')
    .should('have.text', "IGHV3-13*05 1/55/16 J6*02")

  cy.get('#menuCloneNot_full_sequence')
    .click({force: true})
  cy.get('#listElem_1 > .nameBox')
    .should('have.text', "IGHV3-9 7/CCCGGA/17 J6*02")
  cy.get('#listElem_30 > .nameBox')
    .should('have.text', "IGHV3-13*05 1/GAGGGGGGCCTCCCTCCACCCCTCTAACCAGTGAAAAGCAAACTGGGCCCAGCGG/16 J6*02")

  cy.get('#menuCloneNot_nucleotide_number')
    .click({force: true})

\end{verbatim}

\question{Try also the options  \com{alleles in clonotype names} : by selecting \com{always}, the clonotype
  V gene is displayed as \com{IGHV3-9*01}. Revert to the default \com{when not
    *01}. This setting, which is the default, allows to have a more condensed
  V(D)J designation that doesn't make the \com{*01} appear (it is implicit).}


\begin{verbatim}
  //////////////////////////////////
  cy.log("// test_B_clones_03_setting_allele")
  cy.get('#menuAlleleNot_never')
    .click({force: true})
  cy.get('#listElem_1 > .nameBox')
    .should('have.text', "IGHV3-9 7/6/17 J6")

  cy.get('#menuAlleleNot_when_not_01')
    .click({force: true})
  cy.get('#listElem_1 > .nameBox')
    .should('have.text', "IGHV3-9 7/6/17 J6*02")

  cy.get('#menuAlleleNot_always')
    .click({force: true})
  cy.get('#listElem_1 > .nameBox')
    .should('have.text', "IGHV3-9*01 7/6/17 J6*02")

  cy.get('#menuAlleleNot_when_not_01')
    .click({force: true})

\end{verbatim}

\subsection{Showing more clonotypes}

By default Vidjil displays, at each time point, the 50 most abundant clonotypes in the grid, and the 10 most abundant clonotypes in the timeline graph.
With five time points, we may therefore have from 50 to 250 clonotypes displayed
depending if the top 50 are always the same or always different or, more
realistically, in-between.
This number can be increased to a maximum of 100 clonotypes by going to the \com{filter} menu and by putting the
slider to its right end.
\question{Notice how the IGH smaller clonotypes percentage (second clonotype displayed in the list) changes. What was its
  initial value? What is it now?}

\reponse{filter set to 50 \fl IGH smaller clonotypes 10.11\,\%\\*
 filter set to 100 \fl IGH smaller clonotypes 8.92\,\%\\*}
\begin{verbatim}
  /////////////////////////////////
  cy.log("// test_B_clones_04_filter_slider")
  // Igh smaller position: 454


  cy.getCloneSize("454").should('have.text', "10.11%")
  cy.getCloneInList('308').should("not.visible")

  cy.getSliderTop().invoke("val", 65)
    .trigger("change", { force: true })
    .click({ force: true })
    // small clone is present in list (after slider increase)

  cy.getCloneInList('308').should("exist")

  cy.getCloneInList('309').should("not.visible") //"Not all real clones are present in list"
  cy.getCloneSize("454").should('have.text', "9.828%")


  cy.getSliderTop().invoke("val", 50)
    .trigger("change", { force: true })
    .click({ force: true })
  cy.getCloneSize("454").should('have.text', "10.11%")
  cy.getCloneInList('308').should("not.visible")


\end{verbatim}

The \textit{smaller clonotypes} correspond to clonotypes that are not displayed
because they are never among the most abundant ones.


\subsection{Tagging and filtering clonotypes}

Consider the most abundant clonotypes in the list:  \texttt{IGHV3-9 7/CCCGGA/17 J6*02} and  \texttt{TRGV10 13//5 JP1}.
Usually we may want to tag them in order to remember and follow them later on.
\question{Click on the star and choose colored tags for these two clonotypes, such as \texttt{clonotype 1} or \texttt{clonotype 2}.
  Notice how the color applies throughout all the views.}

\begin{verbatim}
  /////////////////////////////////////
  cy.log("// test_B_clones_05_tag_biggest_clone")

  // control default color of clones
  cy.getCloneInList(0).should('have.css', 'color', 'rgb(101, 123, 131)')

  // Tag multiple times one clonotype
  cy.changeTagClone(1, 'clonotype_1')
  cy.changeTagClone(4, "clonotype_2")
  cy.changeTagClone(0, "standard")

  cy.selectClone("5") // select other clone to see real color of clone 0
  cy.getCloneInList(0).should('have.css', 'color', 'rgb(38, 139, 210)')


\end{verbatim}

Later you may want to filter clonotypes depending on the tags you have chosen.

\question{In the upper left part, click on the little dark gray square (the
  second coloured square starting from the right). What happens? What if you click again?}

\begin{verbatim}
  /////////////////////////////////
  cy.log("// test_B_clones_06_filter_by_tag")

  cy.get('#listElem_454 > .nameBox')
    .scrollIntoView()

  cy.get('#listElem_1')
    .should("be.visible")

  cy.getCloneInList("1")
  cy.switchTag("clonotype_1", true)

  cy.getCloneInList('1').should("not.be.visible")

  cy.switchTag("clonotype_1", false)
  cy.getCloneInList('1').should("be.visible")


\end{verbatim}

This is a way of filtering some clonotypes. This may be useful when we want to
focus on some specific clonotypes. Some other way of doing so are to filter them by
their gene names, by their DNA sequences, 
by their locus,
or directly by hidding/focusing some clonotypes.
\question{In the search box,
  enter \texttt{GGAGTCGGGG} and validate with \texttt{Enter}.  How many sequences are
  left?}


\begin{verbatim}
  //////////////////////////////////
  cy.log("// test_B_clones_06b_filter_by_tag")
  // TODO

\end{verbatim}

Note that the search is performed both on the forward and the reverse strand.
\question{Check that by searching for the reverse complement of the
  sequence: \texttt{CCCCGACTCC}. Do you find the same results as previously?}

You can compose several filters:
The menu filter shows the list of currently active filters.

\question{How can you cancel this filter and view again all the clonotypes?}
\reponse{In the filter menu, you can remove filters one by one. You can also remove the text filter by clicking on the cross above the clonotype list.}
\begin{verbatim}
  //////////////////////////////////////
  cy.log("// test_B_clones_07_filter_search_area")

  // cy.getCloneInList('0').should("be.visible")
  cy.getCloneInList('1').should("be.visible")
  cy.getCloneInList('454').should("be.visible")


  cy.filterSearch('CCCCGACTCC')

  // cy.getCloneInList('0').should("not.be.visible")
  cy.getCloneInList('1').should("not.be.visible")
  cy.getCloneInList('454').should("not.be.visible")

  //clones that should stay present (19 clones, test somes)
  cy.getCloneInList('6').scrollIntoView().should("be.visible")
  cy.getCloneInList('142').scrollIntoView().should("be.visible")
  cy.getCloneInList('51').scrollIntoView().should("be.visible")
  cy.getCloneInList('52').scrollIntoView().should("be.visible")
  cy.getCloneInList('16').scrollIntoView().should("be.visible")

  cy.filterSearchReset()
  cy.getCloneInList('1').scrollIntoView().should("be.visible")
  cy.getCloneInList('454').scrollIntoView().should("be.visible")


\end{verbatim}  
\bigskip

Another solution to tag a specific clonotype is to rename it.
\question{Double click on the name of a clonotype (in the list of clonotypes) and
  choose another name (\textit{e.g.} interesting clonotype) and validate using
  \texttt{Enter}.}
\begin{verbatim}
  /////////////////////////////////////
  cy.log("// test_B_clones_08_change_clone_name")

  cy.getCloneInList('1')
    .dblclick()
  cy.get('#new_name')
    .type("interesting clone")
  cy.get('#btnSave').click()

  cy.getCloneInList('1')
    .should('have.text', "interesting clone")

\end{verbatim}

\bigskip

After this rename, you can see that the clonotype is still selected.
\question{Click on several clonotypes by holding the \texttt{Ctrl} key to select
  more. Each time you add a new clonotype to the selection, its sequence
  is added in the bottom part.}
\begin{verbatim}
  //////////////////////////////////
  cy.log("// test_B_clones_09_multiselection")
    cy.changePreset("visu", "read length distribution")
    // to verify correct selection, We will look in segmenter the presence if clone entrie
    // Maybe another method could be more acurate

    cy.selectClone("1")
    cy.getCloneInSegmenter('0').should("not.be.visible") //Clone 0 should not be present anymore in segmenter
    cy.getCloneInList('1').should("be.visible") // Correct selection of clone 1 after second click in scatterplot

    cy.selectCloneMulti([0, 1])
    cy.getCloneInSegmenter(0)
    cy.getCloneInSegmenter(1)

\end{verbatim}

\question{How many clonotypes are selected? How many reads do those clonotypes
  represent?}
\begin{verbatim}
  //////////////////////////////////////////////
  cy.log("// test_B_clones_10_size_if_multiple_selection")
    cy.selectCloneMulti([1, 4])
    cy.get('.stats_content').should('have.text', '2 clonotypes, 364 027 reads (18.51%) ')


\end{verbatim}

\question{ Notice the star at the right of the screen, near the number
  of reads. You can also tag clonotypes using this icon. In that way, you will be able to tag
  all the selected clonotypes at once.}
\begin{verbatim}
  ////////////////////////////////
  cy.log("// test_B_clones_11_multiple_tag")
    cy.selectCloneMulti([7, 11, 13])
    cy.get('#tag_icon__multiple').click()
    cy.update_icon()
    cy.get('#tagElem_custom_2').click()
    cy.update_icon()
    cy.selectClone(0)

    cy.getCloneInList(5).should('have.css', 'color', 'rgb(101, 123, 131)') // clone 5 still have default color
    cy.getCloneInList(7).should('have.css', 'color', 'rgb(55, 145, 73)') // clone 7 have also changed color
    cy.getCloneInList(11).should('have.css', 'color', 'rgb(55, 145, 73)') // clone 11 have also changed color
    cy.getCloneInList(13).should('have.css', 'color', 'rgb(55, 145, 73)') // clone 13 have also changed color


\end{verbatim}

\question{When you want to focus on the selected clonotypes, you can click on the
  focus button (eye icon) on the right, next to the number of selected clonotypes.
  This feature is useful when you want to analyse some clonotypes more thoroughly
  without being annoyed by other clonotypes.}
\begin{verbatim}
  /////////////////////////
  cy.log("// test_B_clones_12_focus")
    cy.getCloneInList(1).should("be.visible")
    cy.getCloneInList(7).should("be.visible")
    cy.getCloneInList(11).should("be.visible")

    cy.selectCloneMulti([1, 7, 11])
    cy.focusOnSelection()

    cy.getCloneInList(1).should("be.visible")
    cy.getCloneInList(7).should("be.visible")
    cy.getCloneInList(11).should("be.visible")
    cy.getCloneInList(5).should("not.be.visible")


\end{verbatim}

\question{To remove this focus, click on the cross next to the search box,
  above the list or on the corresponding line in the filter menu.}
\begin{verbatim}
  /////////////////////////////////
  cy.log("// test_B_clones_13_remove_filter")
    cy.resetAllFilter()
    cy.getCloneInList(1).should("be.visible")
    cy.getCloneInList(7).should("be.visible")
    cy.getCloneInList(11).should("be.visible")
    cy.getCloneInList(5).should("be.visible")


\end{verbatim}

\question{To unselect them all, you can click in an empty area on the top or
  bottom plot.}
\begin{verbatim}
  ///////////////////////////////////////////////////
  cy.log("// test_B_clones_14_unselect_by_click_in_empty_zone")
    // TODO


\end{verbatim}

Sometimes, one wants to hide noisy or unrelated clonotypes.

\question{Select a clonotype or several clonotypes and click on the \com{hide} button (slashed eye icon), near the \com{focus} button. Show again these
  clonotypes by clicking on the cross in the corresponding line in filter menu.}
\begin{verbatim}
  /////////////////////////////////
  cy.log("// test_B_clones_15_hide_selected")
    // test that clone are present before focus
    cy.getCloneInList(1).should("be.visible")
    cy.getCloneInList(7).should("be.visible")
    cy.getCloneInList(11).should("be.visible")

    // select somme clones (7, 11, 13)
    cy.selectCloneMulti([7, 11])
    cy.hideSelection()
    cy.getCloneInList(1).should("be.visible")
    cy.getCloneInList(7).should("not.be.visible")
    cy.getCloneInList(11).should("not.be.visible")
    
    // test that clone are present after resetting filtering
    cy.resetAllFilter()
    cy.getCloneInList(1).should("be.visible")
    cy.getCloneInList(7).should("be.visible")
    cy.getCloneInList(11).should("be.visible")


\end{verbatim}

% Another way to hide clonotypesis to assign is to change the tag of it as ``standard (niose)`` and choose to uncheck this tag by clicking on the corresponding tile on the list of tiles at the informatons panel to switch them from a visible state to a filter one.
%%% Voir ci-dessus, déjà mis

It is also possible to filter samples that do not contain a clonotype.  When you
have lots of samples it helps to keep the sample of interest.  Here the number
of sample is quite limited, so the feature may appear less useful.

\question{Click on the \com{IGHV3-11 / IGHJ6} clonotype in the last sample, whose
  abundance is around 10\,\%.
  Then go in the menu at the upper-right corner of the graph (where \com{5/5}
  is written) and select \com{focus on selected clonotypes}.
}
\begin{verbatim}
  ////////////////////////////////////
  cy.log("// test_B_clones_16_graphmenu_filter")
    cy.getCloneInList(5).should("be.visible")
    cy.get('#visu2_title').should("have.text", "5 / 5")
    cy.get('#time0').should("be.visible")

    cy.get('#visu2_listElem_check_0').should('be.checked')
    cy.get('#visu2_listElem_check_2').should('be.checked')

    cy.selectClone(5)
    cy.get('#visu2_listElem_hideNotShare').click({force: true})

    cy.get('#visu2_title').should("have.text", "3 / 5")
    cy.get('#visu2_listElem_check_0').should('not.be.checked')
    cy.get('#visu2_listElem_check_2').should('be.checked')


\end{verbatim}

By selecting this, the samples where this clonotype doesn't appear are hidden.
This is useful for instance to assess the contamination among dozens of
samples.

\question{You can go back to the previous view by returning into the menu and
  clicking on \com{show all}. Notice also how in the menu you can select the
  samples to be shown.}
\begin{verbatim}
  //////////////////////////////////////
  cy.log("// test_B_clones_17_graphmenu_activate")
    cy.get('#visu2_title').should("have.text", "3 / 5") //Ratio show is correct after focus

    cy.get('#visu2_listElem_check_0').click({force: true})
    cy.update_icon()
    cy.get('#visu2_listElem_check_0').should('be.checked')
    cy.get('#time0').should("be.visible")    // Test "first sample is NOT present in timeline after click"
    cy.get('#visu2_title').should("have.text", "4 / 5") //Ratio show is correct after checkbox clicking

    cy.get('#visu2_listElem_showAll').click({force: true})
    cy.get('#visu2_title').should("have.text", "5 / 5") //Ratio show is correct after checkbox clicking

    cy.get('#visu2_listElem_check_0').should('be.checked')
    cy.get('#visu2_listElem_check_1').should('be.checked')
    cy.get('#visu2_listElem_check_2').should('be.checked')


\end{verbatim}

\section{Analysing clonotype populations}

\subsection{Clustering clonotypes through inspection of their sequences}

The first thing to be done is to see if some clonotypes should be clustered (because
of sequencing or PCR errors for instance). This step could be automatized
but, in any case, the automatic clustering would need to be checked by an expert
eye.

By default in the bottom plot (the \textit{grid}), the clonotypes
  are displayed according to their V and J genes (or more generally to their
  5' and 3' genes). 

\question{Identify in the grid the clonotypes with an
  \textit{IGHV-3-13}~\textit{IGHJ6} recombination and select them
  all. You can do so either by holding \texttt{Ctrl} or by drawing a rectangle around the clonotypes while
  maintaining down the left button of the mouse.}
\begin{verbatim}
  /////////////////////////////////////////////////////
  cy.log("// test_B_clones_18_select_mulitple_ighv3_13_clonotype")
    cy.selectClone(0)
    cy.get("body").type(0)

    cy.get('#visu_id_label_x_IGHV3-13').click()
    cy.update_icon()

    // control that these clonotype is selected and visible in semgenter
    cy.getCloneInSegmenter(0).should("not.be.visible") // should not be present
    cy.getCloneInSegmenter(30)
    cy.getCloneInSegmenter(43)
    cy.getCloneInSegmenter(47)


\end{verbatim}

The sequences of the clonotypes now appear in the bottom part of the browser (the
\textit{sequence panel}). If many clonotypes are selected you can view more sequences
by clicking on the \textit{flat up arrow} in the middle of menu bar of sequence panel.
 
Then, the sequences in the sequence panel can be visually compared but you can also align
them to see more easily their similarities.


\question{Click on the \com{Align} button on the left-hand side. The differences are
emphasized in bold.}
\begin{verbatim}
  /////////////////////////
  cy.log("// test_B_clones_19_align")
    // TODO; align don't available without server


\end{verbatim}

Now it is the user's expertise to determine if sequences are sufficiently
similar, depending on her or his specific question. 
An efficiant hint to do so is to show the quality of each base of the Vidjil windows (menu sequence feature/quality). 
% dans l'exemple, on a des séquence très similaire, à une mutation près. Pourtant, l'id est fortement décallé (~20nt)
We can easily consider that sequence with a mismtach but with a poor quality at this position is a sequencing artifact and could be considered as identical of the support clonotype.
If some sequences don't appear to be similar enough, you can remove
them from the sequence panel by clicking on the cross in front of the sequence in
the sequence panel.
\question{Remove all the sequences that are not similar enough with the first
  one.}
\begin{verbatim}
  ///////////////////////////////////////////
  cy.log("// test_B_clones_20_unselect_from_segmenter")
    cy.removeCloneInSegmenter(47) // automatic assert not visible


\end{verbatim}

Now all the sequences in the sequence panel should be highly similar. All their
differences could be due to sequencing or PCR errors.
These artifacts (mutations, homopolymers, insertions, deletions)
depend on the sequencer and the PCR technique.

\question{Cluster all those clonotypes in a single clonotype by clicking on the ``Cluster''
  button, next to the \com{Align} button.}
\begin{verbatim}
  //////////////////////////////////
  cy.log("// test_B_clones_21_click_on_merge")
    cy.getCloneInSegmenter(30).should("be.visible")
    cy.getCloneInSegmenter(43).should("be.visible")
    cy.get('#cluster').click()
    cy.update_icon()

    cy.getCloneInSegmenter(30).should("be.visible")
    cy.getCloneInSegmenter(43).should("not.be.visible")


\end{verbatim}

All the clustered sequences now appear within a same clonotype. That can be seen
in the list: the clonotype which hosts the subclonotypes appears with a $+$ on its
left. You can click on the $+$ to see the subclonotypes that have been clustered in
the main one.
\question{Click on the $+$ and observe the changes in the grid.}
\begin{verbatim}
  ///////////////////
  cy.log("// test_B_clones_22")
    // TODO


\end{verbatim}

As you may have noticed the subclonotypes appear again in the grid. You can
compare their sequences again if you'd like (for example to double check that
you were right to cluster them). You can also remove some subclonotypes from the
cluster by clicking on the cross at their left in the list.
\question{For the sake of the exercise, remove the last clonotype of the cluster.}
\begin{verbatim}
  ///////////////////
  cy.log("// test_B_clones_23")
    // TODO


\end{verbatim}

\question{%
%For the next step, choose the preset \com{V distribution} (keyboard shortcut \com{5}).
% On n'a pas encore parlé ici des presets. 
Open the \com{cluster} menu at the top of the page, and choose \com{cluster by V/5}. What happened ? There are now two clonotypes with TRGV2. Why ?}
\reponse{These two clonotypes have been clustered on their V gene, but they differ by their allele. You can verify this by changing scatterplot preset by `V/J (allele)`.}
%%  Confirm this by changing the x axis into ``V allele``.
%%% -> Problème, on n'a pas encore parlé des axes à cet endroit.
\begin{verbatim}
  ////////////////////////////////
  cy.log("// test_B_clones_24_cluster_by_V")
    cy.getCloneInList(1).should("not.have.text", "IGHV3-9")
    cy.get('#clusterBox_1 > .icon-plus').should("not.exist")

    cy.get("#clusterBy_5").click({force: true})
    cy.update_icon()

    cy.getCloneInList(1).should("have.text", "IGHV3-9")
    cy.get('#clusterBox_1 > .icon-plus').should("exist")
    cy.getCloneInList(4).should("have.text", "TRGV10")
    cy.get('#clusterBox_4 > .icon-plus').should("exist")


\end{verbatim}

\question{In the \com{cluster} menu, select  \com{revert to previous clusters} to undo these clusterings.}
\begin{verbatim}
  ///////////////////
  cy.log("// test_B_clones_25")
    cy.get('#cluster_break_all').click({force: true})
    cy.update_icon()
    cy.getCloneInList(1).should("have.text", "interesting clone")
    cy.get('#clusterBox_1 > .icon-plus').should("not.exist")
    cy.getCloneInList(4).should("have.text", "TRGV10 13//5 JP1")
    cy.get('#clusterBox_4 > .icon-plus').should("not.exist")


\end{verbatim}

\subsection{Show and use primers information}

You can show primer positions on the sequence according to predefined primer sets 
(Biomed-2 and EuroClonality-NGS by default), and display Genescan-like graph 
using the distance between these positions.
When the primers are not found in the consensus sequence, their position is
extrapolated from the germline sequence.

\question{Select the preset \textit{Primers gap} and observe the starting position of each clonotype.}
\question{Open the settings menu and select the primer set \textit{Biomed2}.\label{q:biomed}}


\begin{verbatim}
\end{verbatim}

This step is computationally intensive and apply on each clonotype of the analysis.
More samples you have, more time it will take to finish.

\question{You should now see a Genescan-like plot.}

This plot show only top clontypes, limited to the top 100 by default.

\question{Change the color by value with \textit{Locus}. What do you see ? }

\question{Change the primer set to \textit{Euroclonality-NGS}. What happens ? }

\begin{verbatim}
\end{verbatim}

\reponse{Remark that some clonotypes become \textit{undefined}. 
This happens for some particular segment, as \textit{IGH J6} .}


\subsection{Other metrics and analysis on the clonotypes}

As a proxy to sequence similarity we used the V and J genes, however there are
other ways to assess sequence similarity that may be more pertinent.
Moreover you may want to plot other metrics of the lymphocyte population.
%
For instance we can choose to plot the V genes versus the length of the N
insertions.
\question{Go to the \com{plot} menu (in the upper left corner of the grid),
  and in the preset box choose \com{V/N length}.}
\begin{verbatim}
  ///////////////////////////////////////
  cy.log("// test_B_clones_26_change_scatter_axis")
    // change preset, first time manually as user can do (don't use $b.scatterplot_select_preset this time)
    cy.changePreset("visu", "V/N length")

    cy.get('#visu_id_legend_x').should("have.text", "V/5' gene")
    cy.get('#visu_id_legend_y').should("have.text", "N length")

    cy.get('#visu_id_label_x_IGHV1-2').should("exist").should("be.visible")
    cy.get('#visu_id_label_y_\\?').should("exist").should("be.visible")


\end{verbatim}

Then you can continue aligning and clustering clonotypes if necessary.

\question{You can also try the preset \com{read length/GC content}
  which tends to separate quite nicely the distinct clonotypes.}
\begin{verbatim}
  ///////////////////////////////////
  cy.log("// test_B_clones_27_change_preset_2")
    cy.changePreset("visu", "read length / GC content")

    cy.get('#visu_id_legend_x').should("have.text", "Reads length")
    cy.get('#visu_id_legend_y').should("have.text", "GC content")

    cy.get('#visu_id_label_x_\\?').should("exist").should("be.visible")


\end{verbatim}

Note that you can choose any axis to be plotted: just go the \com{plot} menu and
select any value you would like for the $x$ axis and for the $y$ axis.
For bar charts, the box sizes always relates to the clonotype size,
and the $y$ axis selects the order of the boxes sharing a same $x$).

%% \item Regarder les stats disponibles, mettre n°7 (taille des reads)

\question{In the \com{plot} menu, switch between the ``bubble plot'' and the ``bar plot''.
In the bar plot mode, pass the mouse over the bars: What happens?}
% on cherche a faire dire quoi ? Le highligth ? Dans ce cas est-il plus simple de le dire directement ? ou en tout cas sans le rapport avec le bar plot.
\begin{verbatim}
  ///////////////////////////////////////////
  cy.log("// test_B_clones_28_switch_scatterplot_mode")

    cy.selectClone(0)
    cy.get("body").type("0")
    cy.update_icon()

    cy.get('#visu_id_legend_x').should("have.text", "V/5' gene")
    cy.get('#visu_id_legend_y').should("have.text", "J/3' gene")

    cy.get("#visu_bar").click({force: true})
    cy.update_icon()

    cy.get('#visu_id_legend_x').should("have.text", "V/5' gene") // Correct legend for axe X after switch in bar mode
    cy.get('#visu_id_legend_y').should("have.text", "Size")      // Correct legend for axe Y after switch in bar mode

    cy.get("#visu_plot").click({force: true})
    cy.update_icon()
    cy.get('#visu_id_legend_x').should("have.text", "V/5' gene") // Correct legend for axe X after switch back in bubble mode"
    cy.get('#visu_id_legend_y').should("have.text", "J/3' gene") // Correct legend for axe Y after switch back in bubble mode"


\end{verbatim}

% Another possibility is to request Vidjil to compute the similarity between
% clones.
% \question{Now select the preset \com{plot by similarity} or even \com{plot
%   similarity by locus} to plot similarity for the current locus (beware: this
% may take some time).}
% Now the most similar clones should be close together. However note that it is
% theoretically impossible to achieve such a representation in 2 dimensions. So
% it is possible that two dissimilar clones are close together or, conversely,
% that two similar clones are far apart.

\question{Press the keys \texttt{0} to \texttt{9} on the numeric keypad. What happens ?}
\reponse{This switches the grid preset. Use shift + number to access to presets 10-19.}
\begin{verbatim}
  ///////////////////////////////////
  cy.log("// test_B_clones_29_shortcut_preset")

    cy.get("body").type(0)
    cy.update_icon()
    cy.get('#visu_id_legend_x').should("have.text", "V/5' gene") // Correct legend for axe X for preset 0
    cy.get('#visu_id_legend_y').should("have.text", "J/3' gene") // Correct legend for axe Y for preset 0
    // cy.get("#visu_id_label_x_IGHV1-2") // scatterplot_legend X at init; IGHV1-2
    // cy.get("#visu_id_label_y_IGHJ1")   // scatterplot_legend Y at init; IGHJ-1

    cy.get("body").type(1)
    cy.update_icon()
    cy.get('#visu_id_legend_x').should("have.text", "V/5' allele") // Correct legend for axe X for preset 1
    cy.get('#visu_id_legend_y').should("have.text", "J/3' allele") // Correct legend for axe Y for preset 1
    // cy.get("#visu_id_label_x_IGHV1-2*04") // scatterplot_legend X with shortcut/preset XXX is: IGHV1-2
    // cy.get("#visu_id_label_y_IGHJ1")      // scatterplot_legend Y with shortcut/preset XXX is: IGHJ-1


    cy.get("body").type(2)
    cy.update_icon()
    cy.get('#visu_id_legend_x').should("have.text", "V/5' gene") // Correct legend for axe X for preset 2
    cy.get('#visu_id_legend_y').should("have.text", "N length")  // Correct legend for axe Y for preset 2
    // cy.get("#visu_id_label_x_IGHV1-2") // scatterplot_legend X with shortcut/preset XXX is: IGHV1-2
    // cy.get("#visu_id_label_y_\\?")     // scatterplot_legend Y with shortcut/preset XXX is: IGHJ-1

    cy.get("body").type(3)
    cy.update_icon()
    cy.get('#visu_id_legend_x').should("have.text", "Reads length") // Correct legend for axe X for preset 3
    cy.get('#visu_id_legend_y').should("have.text", "Locus")        // Correct legend for axe Y for preset 3
    // cy.get("#visu_id_label_x_20")  // scatterplot_legend X with shortcut/preset XXX is: IGHV1-2
    // cy.get("#visu_id_label_y_IGH") // scatterplot_legend Y with shortcut/preset XXX is: IGHJ-1

    cy.get("body").type(4)
    cy.update_icon()
    cy.get('#visu_id_legend_x').should("have.text", "Reads length") // Correct legend for axe X for preset 4
    cy.get('#visu_id_legend_y').should("have.text", "Size")         // Correct legend for axe Y for preset 4
    // cy.get("#visu_id_label_x_20")   // scatterplot_legend X with shortcut/preset XXX is: IGHV1-2
    // cy.get("#visu_id_label_y_20\%") // scatterplot_legend Y with shortcut/preset XXX is: IGHJ-1

    cy.get("body").type(5)
    cy.update_icon()
    cy.get('#visu_id_legend_x').should("have.text", "V/5' gene") // Correct legend for axe X for preset 5
    cy.get('#visu_id_legend_y').should("have.text", "Size")      // Correct legend for axe Y for preset 5
    // cy.get("#visu_id_label_x_IGHV1-2") //scatterplot_legend X with shortcut/preset XXX is: IGHV1-2
    // cy.get("#visu_id_label_y_16\%")    //scatterplot_legend Y with shortcut/preset XXX is: IGHJ-1

    cy.get("body").type(6)
    cy.update_icon()
    cy.get('#visu_id_legend_x').should("have.text", "N length") // Correct legend for axe X for preset 6
    cy.get('#visu_id_legend_y').should("have.text", "Size")     // Correct legend for axe Y for preset 6
    // cy.get("#visu_id_label_x_\\?")  // scatterplot_legend X with shortcut/preset XXX is: IGHV1-2
    // cy.get("#visu_id_label_y_55\%") // scatterplot_legend Y with shortcut/preset XXX is: IGHJ-1

    cy.get("body").type(7)
    cy.update_icon()
    cy.get('#visu_id_legend_x').should("have.text", "CDR3 length") // Correct legend for axe X for preset 7
    cy.get('#visu_id_legend_y').should("have.text", "Size")        // Correct legend for axe Y for preset 7
    // cy.get("#visu_id_label_x_") // scatterplot_legend X with shortcut/preset XXX is: IGHV1-2
    // cy.get("#visu_id_label_y_") // scatterplot_legend Y with shortcut/preset XXX is: IGHJ-1

    cy.get("body").type(8)
    cy.update_icon()
    cy.get('#visu_id_legend_x').should("have.text", "J/3' gene") // Correct legend for axe X for preset 8
    cy.get('#visu_id_legend_y').should("have.text", "Size")      // Correct legend for axe Y for preset 8
    // cy.get("#visu_id_label_x_") // scatterplot_legend X with shortcut/preset XXX is: IGHV1-2
    // cy.get("#visu_id_label_y_") // scatterplot_legend Y with shortcut/preset XXX is: IGHJ-1

    cy.get("body").type(9)
    cy.update_icon()
    cy.get('#visu_id_legend_x').should("have.text", "Size")         // Correct legend for axe X for preset 9
    cy.get('#visu_id_legend_y').should("have.text", "Size (other)") // Correct legend for axe Y for preset 9
    // cy.get("#visu_id_label_x_") // scatterplot_legend X with shortcut/preset XXX is: IGHV1-2
    // cy.get("#visu_id_label_y_") // scatterplot_legend Y with shortcut/preset XXX is: IGHJ-1

    cy.get("body").type(0)
    cy.update_icon()
    cy.get('#visu_id_legend_x').should("have.text", "V/5' gene") // Correct legend for axe X for preset 0
    cy.get('#visu_id_legend_y').should("have.text", "J/3' gene") // Correct legend for axe Y for preset 0
    // cy.get("#visu_id_label_x_IGHV1-2")// scatterplot_legend X with shortcut/preset XXX is: IGHV1-2
    // cy.get("#visu_id_label_y_\\?")    // scatterplot_legend Y with shortcut/preset XXX is: IGHJ-1

\end{verbatim}

There is still a feature to help you analyse your data that we have not
explored yet.
You can change the colors to make it represent some variables of interest
with the \com{color by} menu.
\question{First choose the preset \com{V/J (genes)} and
  then color by \com{N length} (in the box at the top of the screen).}
\begin{verbatim}
  //////////////////////////////
  cy.log("// test_B_clones_30_color_mode")

    cy.changePreset("visu", "V/J (genes)")
    cy.get("#visu_id_label_x_IGHV1-2") // scatter plot legend X is correct after preset change
    cy.get("#visu_id_label_y_IGHJ1")   // scatter plot legend Y is correct after preset change

    // control color state at init
    cy.get('.gradient').should('not.exist')
    cy.changeColorby("N length")
    cy.get('.gradient').should('exist')

    // control
    cy.getCloneInList(1).should('have.css', 'color', 'rgb(0, 189, 225)')
    cy.getCloneInList(4).should('have.css', 'color', 'rgb(0, 173, 251)')


\end{verbatim}
  
\marginpar{We apologize to color blinds: the colors are not yet color-blind friendly.}Clonotypes that are close on the grid with similar colors are likely to
be similar.
% hum hum, voir si c'est bien vrai. Juste avec le Nlength ?

\question{Choose now the preset \com{CDR3 length distribution} and
  then color by \com{size}.
  See that the color tiles in the info part (upper right) change to show the color key.}
Discrete axes (as tags) are shown with a selector.
Some axes with many values (such as V and J genes) will not be displayed here.
Continuous axes are shown with a radient color.
\begin{verbatim}
  //////////////////////////////////
  cy.log("// test_B_clones_31_color_gradient")

    cy.changePreset("visu", "CDR3 length distribution")
    cy.changeColorby("Size")
    cy.update_icon()

    cy.get('.gradient').should('exist')


\end{verbatim}

\question{ Instead of coloring by clonotype size, you could also color by
  \com{clonotype}. When coloring by \com{clonotype}, each clonotype has a random color. Thus in
  a bar plot, it is a convenient color mode to see the peaks that are due to a
  single clonotype or to several clonotypes.
  However clonotypes may be very similar. Another option is to color by
  \com{CDR3}. In such case all clonotypes with the same CDR3 will have the same
  color (note that, due to a lack of available colors two different CDR3s
  could share the same color just by chance).}
  % CDR3 par AA ? sinon comment puisqu'il doit y a voir au moins une variation...
  % A partir de quand ds ce cas se retrouve-t-on avec une couleur completement différente ?
\begin{verbatim}
  //////////////////////////////////
  cy.log("// test_B_clones_32_color_by_clone")
    cy.changeColorby("Clonotype")
    cy.update_icon()

    cy.selectClone(0)
    cy.getCloneInList(1).should('have.css', 'color', 'rgb(153, 153, 61)')
    cy.getCloneInList(4).should('have.css', 'color', 'rgb(61, 61, 153)')


\end{verbatim}


Using those different features you should be able to analyse how similar your
sequences are, and potentially you could cluster them if you'd like or tag them.

\question{
  Select the most abundant clonotype. It now appears in the sequence panel.
  Now we would like to compare the sequence with the germline genes.
  We can add the germline genes to the sequence panel by going
  to the \com{import/export} menu and by clicking on \com{add germline genes}.
  % toujours noté experimental d'ailleur
  Now we can click on the \com{align} button to see the alignment between the
  genes and the sequence. Mutations can be identified and silent mutations are
  displayed with a double border in blue.
}
\begin{verbatim}
  //////////////////////////////////
  cy.log("// test_B_clones_33_align_germline")

    // sequence germline is NOT present in the segmenter
    cy.get('#seqIGHJ6\*02 > .sequence-holder > .seq-fixed > .nameBox').should("not.exist")

    cy.selectClone(1)
    cy.update_icon()
    cy.getCloneInSegmenter(1)
    cy.get("#export_add_germline").click({force: true})

    cy.get('#seqIGHJ6\\*02 > .sequence-holder > .seq-fixed > .nameBox').should("exist")
    // todo, align; mutation; not available without server side


\end{verbatim}

\bigskip

\textit{This part is specific to samples analyzed with Vidjil-algo.}

Some clonotypes may be less trustable than other ones\dots{} Let's see how to spot them.
\question{In the clonotype list, search clonotypes with an orange warning at the
  right side. Click on the warning. What are the warnings due to?}
\begin{verbatim}
  ///////////////////
  cy.log("// test_B_clones_34")
    cy.selectClone(11)
    cy.get('#clone_infoBox_1  > .icon-warning-1').should("not.exist")
    cy.get('#clone_infoBox_11 > .icon-warning-1').should("exist")

    cy.openCloneInfo("11")

    cy.get('#clone_info_table_11')
      .should("exist")
      //.should("be.visible") // bottom part of table is covered; don't work

    cy.get('#modal_line_title_W53')
      .should("exist")
      .should("be.visible")
    cy.get('#modal_line_value_W53')
      .should("have.text", "Similar to clone #2 - TRGV10*01 13//5 TRGJP1*01")

    cy.closeCloneInfo()


\end{verbatim}

There may have several reasons: 
\begin{itemize}
\item average coverage: in that case the clonal sequence displayed is short
  compared to the reads in the clonotype. This may be the case when too different
  sequences have been put in a clonotype. The value is generally $\geq 80\,\%$.
  % nuancer, toujours valable ? Si merge ou non ?
\item $e$-value: It is a statistical value computed to ensure that
  recombinations have not been spot by chance. This value is generally much
  lower than 1 ($<10^{-5}$).
\item Clonotype similar to another one: In that case Vidjil-algo tells you that
  other clonotypes have the same genes and may be similar
\item Several genes with equal probability: The algorithm found several alleles or genes with the exact same e-value. This may happen when the sequence is too short.
\item Non-recombined sequences: Some known unrecombined sequences are tagged
  so that you can spot them easily. We tag the unrecombined IGHD7-27/IGHJ1
  sequences that may be amplified.
\end{itemize}

You can view those values for any clonotype by clicking the \textit{i} icon on the
right side, in the list of clonotypes.

The existing warnings are listed on \href{https://www.vidjil.org/doc/warnings/}{www.vidjil.org/doc/warnings/}.

\subsection{Show sequence features and information}

When you select a clonotype, you first see, in the aligner, the DNA sequence and its V, (D), and J segments.
Next to the "align" button, several menus allow to display more features or information about the sequences. We will explore those menus.
 
\question{Select the first clonotypes both in IGH and TRG. Then in the \textit{Data columns} menu (the first on the left after the align button), click on \textit{productivity}.}

\question{Now, click on the spin icon.}

This queries external tools to process the selected sequences (as for example IMGT/V-QUEST)
and, after a short time, will add the results to the clonotypes (as for example productivity computed by IMGT/V-QUEST, which is displayed by default).

Contact us if you need to query other software.


\question{Open the \textit{sequence feature} menu (the one before the last button). Click on "CDR3" and "quality".}

In the aligner, there are now two new overlays on sequences. 
Above the sequence, the CDR3 is shown (you may have to move the horizontal scrollbar to view the CDR3).
Under the sequence, the base quality of each position is shown, with a gradient color, from green (high quality) to red (low quality). The quality is shown for the window centered on the CDR3.

When aligning sequences, the quality color may help you to discriminate between potential biological mutations and sequencing errors (note that the current dataset is quite old and was sequenced with Ion Torrent, now the quality is usually much better).


Note that some features, such as \textit{primers} or \textit{IMGT productivity}, are available only after some additional computations.



\begin{verbatim}
\end{verbatim}

\subsection{Analysing recombinations from several loci}

First make sure to come back to the preset \com{V/J (genes)} in the \com{plot} menu.

If you want to focus on specific locus, you can click on the locus name in
the upper left part. One click will make the locus disappear, another one will
make it appear again.
If you hold the \texttt{Shift} key (the one which is usually above the left
\texttt{Ctrl} key) while clicking it will hide all the loci but the one you
clicked on.

\question{Click on \com{IGH}, while holding the \texttt{Shift} key. Now what is the
  number of analyzed reads? Why did it change?}
\begin{verbatim}
  ////////////////////////////////
  cy.log("// test_B_clones_35_locus_switch")
    cy.get('#toogleLocusSystemBox_TRG').should('not.have.class', 'unchecked')
    cy.get('#toogleLocusSystemBox_IGH').should('not.have.class', 'unchecked')

    cy.get('#toogleLocusSystemBox_IGH').click({shiftKey: true})
    cy.update_icon()

    cy.get('#toogleLocusSystemBox_TRG').should('have.class', 'unchecked')     // locus TRG NOT present in info panel after switch
    cy.get('#toogleLocusSystemBox_IGH').should('not.have.class', 'unchecked') // locus IGH present in info panel after switch


\end{verbatim}

\question{Now click on \com{TRG}, to filter it in again.}
\begin{verbatim}
  ////////////////////////////////////////
  cy.log("// test_B_clones_36_activate_again_locus")
    cy.get('#toogleLocusSystemBox_TRG').click()
    cy.update_icon()
    cy.get('#toogleLocusSystemBox_TRG').should('not.have.class', 'unchecked')
    cy.get('#toogleLocusSystemBox_IGH').should('not.have.class', 'unchecked')

\end{verbatim}

\question{Press on the \texttt{g} key. What happens? Now, press on the
  \texttt{h} key. Press on the \texttt{g} again (you can do that anytime you
  like :)). Let's stick to the TRG locus.}
\begin{verbatim}
  ////////////////////////////
  cy.log("// test_B_clones_37_shortcut")
    cy.get("body").type(0) // return to a grap with locus 
    cy.update_icon()
    cy.get('#visu_id_label_x_IGHV1-2').should("exist").should("be.visible")
    cy.get('#visu_id_label_y_IGHJ1').should("exist").should("be.visible")

    cy.get("body").type("g") // locus TRG
    cy.update_icon()
    cy.get('#visu_id_label_x_TRGV2').should("exist").should("be.visible")
    cy.get('#visu_id_label_y_TRGJ1').should("exist").should("be.visible")

    cy.get("body").type("h") // locus IGH
    cy.update_icon()
    cy.get('#visu_id_label_x_IGHV1-2').should("exist").should("be.visible")
    cy.get('#visu_id_label_y_IGHJ1').should("exist").should("be.visible")


\end{verbatim}

You can also change the current locus by clicking on the locus name in the
right part of the grid.
\begin{verbatim}
  /////////////////////////////////////////////////////
  cy.log("// test_B_clones_38_switch_locus_by_scatterplot_click")
    cy.get("body").type(0) // locus IGH
    cy.update_icon()
    cy.get('#visu_id_label_x_IGHV1-2').should("exist").should("be.visible") // scatterplot_legend X at init; IGHV1-2
    cy.get('#visu_id_label_y_IGHJ1').should("exist").should("be.visible")   // scatterplot_legend Y at init; IGHJ-1

    cy.get('#visu_id_sp_system_label_TRG > .sp_system').click()
    cy.update_icon()
    cy.get('#visu_id_label_x_TRGV2').should("exist").should("be.visible")
    cy.get('#visu_id_label_y_TRGJ1').should("exist").should("be.visible")


\end{verbatim}

\subsection{Clonotype quantification (using spike-ins)}

Sometimes you may include spike-ins in your sample to allow a more reliable
quantification.
Let us assume that the main clonotype with IGHV-3-9 / IGHJ5 is a spike-in whose
expected concentration is 1\% (.01).

\question{First let's color this clonotype with the \com{standard} tag.}
\begin{verbatim}
  ///////////////////////////////////
  cy.log("// test_B_clones_39_reset_color_tag")
    cy.changeColorby("Tag")
    cy.getCloneInList(0).should('have.css', 'color', 'rgb(38, 139, 210)')


\end{verbatim}

\question{ Now we will set its concentration to 1\% as expected. Click again on
  the star. In the \com{normalize to} field enter \com{1} and click \com{ok}.
  Now, in the graph, this clonotype should correspond to a straight line at 1\%.}
  % on pourrait indiquer dans le champs la valeur en pourcent ou non, ce n'est pas clair 
  % l'iunclur en placeholder ? 
\begin{verbatim}
  /////////////////////////////////
  cy.log("// test_B_clones_40_normalization")
    cy.getCloneSize("1").should('have.text', "9.665%")

    cy.openTagPanelOneClone(1)
    cy.get('#norm_button').type("1{enter}")
    cy.getCloneSize("1").should('have.text', "1.000%")

    // need a refresh of the menu, call by hover
    cy.get('#settings_menu').invoke('trigger', "mouseover")
    cy.get('#settings_menu').invoke('trigger', "mouseout")
    cy.get('#settings_menu').invoke('trigger', "mouseenter")
    cy.get('#settings_menu').invoke('trigger', "mouseleave")
    cy.get('#settings_menu').invoke('show')

    cy.get("#normalizetest1").should("exist") // Form have the input for expected normalization"


\end{verbatim}

\question{ Notice how the concentrations of the other clonotypes have changed
  accordingly.
  You can go to the \com{settings} menu to disable this normalization and to
  go back to the raw concentrations.}
\begin{verbatim}
  /////////////////////////////////////////
  cy.log("// test_B_clones_41_disable_normalization")
    cy.get("#reset_norm").click({force: true})
    cy.getCloneSize("1").should('have.text', "9.665%") //correct size after disabled normalization"

\end{verbatim}

Then you can set expected concentrations for other clonotypes and you are free to
switch between those normalizations.
It is also possible to set up normalization against external data,
contact us if you are interested.

\textit{Please note that we have more advanced features to include your own spike-ins and to take into account spike-ins for different gene families. Contact us if you are interested.}
