\section{Viewing and filtering clones}


\subsection{Looking at a clone}
\begin{verbatim}


  #################################
  ### Viewing and filtering clones
  #################################

\end{verbatim}

Each RepSeq algorithm has its own definition of what a clone is (or, more precisely
a clonotype), and on how to output its sequence and how to assign a V(D)J designation.

In this file, the most abundant clone
is \texttt{IGHV3-9 7/CCCGGA/17 J6*02}.

\question{Select this clone, either by clicking on the list or on the grid.
  How many reads do this clone represent? (see again the bottom part to the right)}

\reponse{The bottom panel display information about currently selected clones \fl 189 991 reads (9.665\,\%)}

\begin{verbatim}
  def test_B_clones_01_major
    # major clone of sample 0 is clone 1 
    clone = $b.clone_in_list("1").click
    cinfo = $b.clone_info("1")
    assert ( cinfo[:name].text == "IGHV3-9 7/CCCGGA/17 J6*02"), "correct name for first clone"

    stats = $b.statsline
    assert (stats.text.include? '1 clonotype, 189 991 reads (9.665%)'),  "Correct stats, number of reads and percentage: " + stats.text
  end
\end{verbatim}


There are several options to display the V(D)J designation.

\question{In the  \com{settings} menu, under \com{N regions in clonotype names} select \com{length} to show N zones by their length. Revert to the
  default \com{sequence (when short)} setting to show the full N on short sequences.}

\begin{verbatim}
  def test_B_clones_02_setting_n_length
    $b.menu_settings.click
    $b.input(:id => "menuCloneNot_nucleotide_number").click
    $b.update_icon.wait_while(&:present?) # wait update
    cinfo = $b.clone_info("1")
    assert ( cinfo[:name].text == "IGHV3-9 7/6/17 J6*02"), "correct name for first clone (nucleotide_number)"
    cinfo30 = $b.clone_info("30")
    assert ( cinfo30[:name].text == "IGHV3-13*05 1/55/16 J6*02"), "correct name for 30th clone (nucleotide_number)"

    $b.menu_settings.click
    $b.input(:id => "menuCloneNot_short_sequence").click
    $b.update_icon.wait_while(&:present?) # wait update
    cinfo = $b.clone_info("1")
    assert ( cinfo[:name].text == "IGHV3-9 7/CCCGGA/17 J6*02"), "correct name for first clone (short_sequence)"
    cinfo30 = $b.clone_info("30")
    assert ( cinfo30[:name].text == "IGHV3-13*05 1/55/16 J6*02"), "correct name for 30th clone (short_sequence)"

    $b.menu_settings.click
    $b.input(:id => "menuCloneNot_full_sequence").click
    $b.update_icon.wait_while(&:present?) # wait update
    cinfo = $b.clone_info("1")
    assert ( cinfo[:name].text == "IGHV3-9 7/CCCGGA/17 J6*02"), "correct name for first clone (full_sequence)"
    cinfo30 = $b.clone_info("30")
    assert ( cinfo30[:name].text == "IGHV3-13*05 1/GAGGGGGGCCTCCCTCCACCCCTCTAACCAGTGAAAAGCAAACTGGGCCCAGCGG/16 J6*02"), "correct name for 30th clone (full_sequence)"

    # Reset params as nucleotide number
    $b.menu_settings.click
    $b.input(:id => "menuCloneNot_nucleotide_number").click
    $b.update_icon.wait_while(&:present?) # wait update
  end
\end{verbatim}

\question{Try also the options  \com{alleles in clonotype names} : by selecting \com{always}, the clone
  V gene is displayed as \com{IGHV3-9*01}. Revert to the default \com{when not
    *01}. This setting, which is the default, allows to have a more condensed
  V(D)J designation that doesn't make the \com{*01} appear (it is implicit).}


\begin{verbatim}
  def test_B_clones_03_setting_allele
    $b.menu_settings.click
    $b.input(:id => "menuAlleleNot_never").click
    $b.update_icon.wait_while(&:present?) # wait update
    cinfo = $b.clone_info("1")
    assert ( cinfo[:name].text == "IGHV3-9 7/6/17 J6"), "correct name for first clone (menuAlleleNot_never): " + cinfo[:name].text

    $b.menu_settings.click
    $b.input(:id => "menuAlleleNot_when_not_01").click
    $b.update_icon.wait_while(&:present?) # wait update
    cinfo = $b.clone_info("1")
    assert ( cinfo[:name].text == "IGHV3-9 7/6/17 J6*02"), "correct name for first clone (menuAlleleNot_when_not_01)"

    $b.menu_settings.click
    $b.input(:id => "menuAlleleNot_always").click
    $b.update_icon.wait_while(&:present?) # wait update
    cinfo = $b.clone_info("1")
    assert ( cinfo[:name].text == "IGHV3-9*01 7/6/17 J6*02"), "correct name for first clone (menuAlleleNot_always)"

  end
\end{verbatim}

\subsection{Showing more clones}

By default Vidjil displays the 50 most abundant clones at each time point.
With five time points, we may therefore have from 50 to 250 clones displayed
depending if the top 50 are always the same or always different or, more
realistically, in-between.
This number can be increased to a maximum of 100 clones by going to the \com{filter} menu and by putting the
slider to its right end.
\question{Notice how the IGH smaller clonotypes percentage (second clone displayed in the list) changes. What was its
  initial value? What is it now?}

\reponse{filter set to 50 \fl IGH smaller clonotypes 10.11\,\%\\*
 filter set to 100 \fl IGH smaller clonotypes 8.92\,\%\\*}
\begin{verbatim}
  def test_B_clones_04_filter_slider
    if not ENV['HEADLESS']
      # Igh small position: 454
      cinfo = $b.clone_info("454")
      assert ( cinfo[:size].text == "10.11%"), "correct name for first clone (short_sequence)"

      ## Change slider value
      assert ( not $b.clone_in_list('308').present?), "small clone is not present in list (before slider increase)"
      slider = $b.menu_item("top_slider")
      slider.click
      $b.menu_filter.hover
      slider.send_keys :arrow_right 
      slider.send_keys :arrow_right
      slider.send_keys :arrow_right
      $b.clone_in_list("0").click
      ### exist in scatterplot
      $b.clone_in_list('308').wait_until(&:present?) # "small clone is present in list (after slider increase)"
      assert ( not $b.clone_in_list('309').present?), "Not all real clones are present in list"

      ## control by size of smaller clone
      assert ( cinfo[:size].text == "9.768%"), "correct name for first clone (short_sequence)"

      ### decrease by slider usage
      slider = $b.menu_item("top_slider")
      slider.click
      $b.menu_filter.hover
      slider.send_keys :arrow_left 
      $b.clone_in_list("0").click
      $b.update_icon.wait_while(&:present?)

      ### exist in scatterplot
      assert ( not $b.clone_in_list('308').present?), "a small clone is hidden if slider decrease"
      assert ( cinfo[:size].text == "10.11%"), "correct name for first clone (short_sequence) after decrease slider value"
    end
  end
\end{verbatim}

The \textit{smaller clonotypes} correspond to clones that are not displayed
because they are never among the most abundant ones.


\subsection{Tagging and filtering clones}

Consider the most abundant clones in the list:  \texttt{IGHV3-9 7/CCCGGA/17 J6*02} and  \texttt{TRGV10 13//5 JP1}.
Usually we may want to tag them in order to remember them later on.
\question{Click on the star and choose colored tags for these two clones, such as \texttt{clonotype 1} or \texttt{clonotype 2}.
  Notice how the color applies throughout all the views.}

\begin{verbatim}
  def test_B_clones_05_tag_biggest_clone
    # control default color of clones
    assert ($b.clone_info('0')[:name].style('color') ==  'rgba(101, 123, 131, 1)' )

    $b.clone_in_list("0").click
    $b.clone_info("1")[:star].click # top IGH
    $b.tag_item('clonotype_1')[:name].click
    $b.clone_info("4")[:star].click # top TRG 
    $b.tag_item('clonotype_2')[:name].click
    $b.clone_info("0")[:star].click ## clone rechute
    $b.tag_item('standard_2')[:name].click
    $b.update_icon.wait_while(&:present?)
    $b.clone_in_list("5").click ## select other clone to see real color of clone 0

    # New color
    assert ($b.clone_info('0')[:name].style('color') ==  'rgba(108, 113, 196, 1)' )
  end
\end{verbatim}

Later you may want to filter clones depending on the tags you have chosen.

\question{In the upper left part, click on the little dark gray square (the
  second coloured square starting from the right). What happens? What if you click again?}

\begin{verbatim}
  def test_B_clones_06_filter_by_tag
    assert ($b.clone_in_list("1").present?), "clone 0 is present at starting test"

    # Click on tag switch; clone 1 will be hidden
    $b.info_colorBy.span(title: 'clonotype 1').click
    $b.update_icon.wait_while(&:present?)
    assert (not $b.clone_in_list("1").present?), "clone 0 is hidden"

    $b.info_colorBy.span(title: 'clonotype 1').click
    $b.update_icon.wait_while(&:present?)
    assert ($b.clone_in_list("1").present?), "clone 0 is present at the end of the test"
  end
\end{verbatim}

This is a way of filtering some clones. This may be useful when we want to
focus on some specific clones. Another way of doing so is to filter them by
their gene names or by their DNA sequences.  
\question{In the search box,
  enter \texttt{GGAGTCGGGG} and validate with \texttt{Enter}.  How many sequences are
  left?}


\begin{verbatim}
  def test_B_clones_06b_filter_by_tag
    # TODO
  end
\end{verbatim}

Note that the search is performed both on the forward and the reverse strand.
\question{Check that by searching for the reverse complement of the
  sequence: \texttt{CCCCGACTCC}. Do you find the same results as previously?}
\question{How can you cancel this filter and view again all the clones?}

\begin{verbatim}
  def test_B_clones_07_filter_search_area
    ## test that clone are present before filtering
    assert ( $b.clone_in_list('0').present?),  ">> clone 0 is present"
    assert ( $b.clone_in_list('1').present?),  ">> clone 1 is present"
    #assert ( $b.clone_in_list('212').present?),  ">> clone 212 is present"
    assert ( $b.clone_in_list('454').present?),  ">> clone 454 is present"

    filter = $b.filter_area
    filter.value = 'CCCCGACTCC'
    $b.send_keys :enter
    $b.update_icon.wait_while(&:present?)
    # clone present before filtering; not after
    assert ( not $b.clone_in_list('0').present?), ">> clone 0 is hidden"
    assert ( not $b.clone_in_list('1').present?), ">> clone 1 is hidden"
    #assert ( not $b.clone_in_list('212').present?), ">> clone 212 is hidden"
    assert ( not $b.clone_in_list('454').present?), ">> clone 454 is hidden"
    # clones that should stay present (19 clones, test somes)
    assert (     $b.clone_in_list('6').present?),   ">> clone 6   is present even with filtering"
    assert (     $b.clone_in_list('142').present?), ">> clone 142 is present even with filtering"
    assert (     $b.clone_in_list('51').present?),  ">> clone 51  is present even with filtering"
    assert (     $b.clone_in_list('52').present?),  ">> clone 52  is present even with filtering"
    assert (     $b.clone_in_list('16').present?),  ">> clone 16  is present even with filtering"

    # tester le nombre de clone présents

    $b.clear_filter.click
    $b.clone_in_scatterplot('1').hover
    ## test that clone are present again after remove filter
    assert ( $b.clone_in_list('0').present?),   ">> clone 0 is present again"
    assert ( $b.clone_in_list('1').present?),   ">> clone 1 is present again"
    #assert ( $b.clone_in_list('212').present?), ">> clone 212 is present again"
    assert ( $b.clone_in_list('454').present?), ">> clone 454 is present again"
  end
\end{verbatim}  
\bigskip

Another solution to tag a specific clone is to rename it.
\question{Double click on the name of a clone (in the list of clones) and
  choose another name (\textit{e.g.} interesting clone) and validate using
  \texttt{Enter}.}
\begin{verbatim}
  def test_B_clones_08_change_clone_name
    $b.clone_in_list("1").click
    cinfo = $b.clone_info("1")
    cinfo[:name].double_click # set mode to allow setting

    $b.clone_name_editor.set 'interesting clone'
    $b.clone_name_saver.click
    $b.update_icon.wait_while(&:present?)
    assert (cinfo[:name].text == 'interesting clone'), " >> clone name ('interesting clone') has changed"
  end
\end{verbatim}

\bigskip

After this rename, you can see that the clone is still selected.
\question{Click on several clones by holding the \texttt{Ctrl} key to select
  more. Each time you add a new clone to the selection, its sequence
  is added in the bottom part.}
\begin{verbatim}
  def test_B_clones_09_multiselection
    $b.menu_filter.click
    $b.update_icon.wait_while(&:present?)
    $b.send_keys 4
    $b.update_icon.wait_while(&:present?)
    # to verify correct selection, We will look in segmenter the presence if clone entrie
    # Maybe another method could be more acurate
    
    $b.clone_in_list("1").click
    $b.update_icon.wait_while(&:present?)
    assert ( not $b.clone_in_segmenter('0').present? ), ">> Another click; Clone 0 should not be present anymore in segmenter"
    assert (     $b.clone_in_segmenter('1').present? ), ">> Another click; Correct selection of clone 1 after second click in scatterplot"
   
    $b.clone_in_list("0").click(:control)
    $b.update_icon.wait_while(&:present?)
    # clone is present in segmenter
    assert ( $b.clone_in_segmenter('0').present? ), ">> ctrl+click; Clone 0 should be present in segmenter"
    assert ( $b.clone_in_segmenter('1').present? ), ">> ctrl+click; Clone 1 should be present in segmenter"
    # As whell than their sequences ? TODO
  end
\end{verbatim}

\question{How many clones are selected? How many reads do those clones
  represent?}
\begin{verbatim}
  def test_B_clones_10_size_if_multiple_selection
    $b.clone_in_list("1").click
    $b.clone_in_list("4").click(:control)
    $b.update_icon.wait_while(&:present?)

    stats = $b.statsline
    assert (stats.text.include? '2 clonotypes, 364 027 reads (18.51%)'),  "Correct stats, number of reads and percentage"
  end
\end{verbatim}

\question{ Notice the star at the the right of the screen, near the number
  of reads. You can also tag clones using this icon. In that way, you will be able to tag
  all the selected clones at once.}
\begin{verbatim}
  def test_B_clones_11_multiple_tag
    $b.clone_in_list("7").click
    $b.clone_in_list("11").click(:control)
    $b.clone_in_list("13").click(:control)
    $b.update_icon.wait_while(&:present?)

    $b.element(:id => "tag_icon__multiple").click
    $b.element(:id => 'tagElem_custom_2').click
    $b.update_icon.wait_while(&:present?)
    $b.clone_in_list("0").click(:control) # select another clone, else last selected still keep default color as "hover"
    assert ($b.clone_info('5')[:name].style('color') ==  'rgba(101, 123, 131, 1)' ) , "clone 5 still have default color"
    assert ($b.clone_info('7')[:name].style('color') ==  'rgba(211, 54, 130, 1)' ) , "clone 7 have also changed color"
    assert ($b.clone_info('11')[:name].style('color') ==  'rgba(211, 54, 130, 1)' ) , "clone 11 have also changed color"
    assert ($b.clone_info('13')[:name].style('color') ==  'rgba(211, 54, 130, 1)' ) , "clone 13 have also changed color"
  end
\end{verbatim}

\question{When you want to focus on the selected clones, you can click on the
  focus link on the right, next to the number of selected clones.
  This feature is useful when you want to analyse some clones more thoroughly
  without being annoyed by other clones.}
\begin{verbatim}
  def test_B_clones_12_focus
    ## test that clone are present before focus
    assert ( $b.clone_in_list('1').present?), ">> clone 1 is present"
    assert ( $b.clone_in_list('7').present?), ">> clone 7 is present"
    assert ( $b.clone_in_list('11').present?), ">> clone 11 is present"
    
    # select somme clones (7, 11, 13)
    $b.clone_in_list("7").click
    $b.clone_in_list("11").click(:control)
    $b.clone_in_list("13").click(:control)
    $b.update_icon.wait_while(&:present?)
    $b.span(:id => "focus_selected").click
    $b.update_icon.wait_while(&:present?)

    ## test that clone are hidden after focus
    assert ( not $b.clone_in_list('1').present?), ">> clone 1 is hidden"
    assert (     $b.clone_in_list('7').present?), ">> clone 7 is present"
    assert (     $b.clone_in_list('11').present?), ">> clone 3 is present"
    assert ( not $b.clone_in_list('5').present?), ">> clone 5 is hidden"
  end
\end{verbatim}

\question{To remove this focus, click on the cross next to the search box,
  above the list.}
\begin{verbatim}
  def test_B_clones_13_remove_filter
    ## test that clone are present after resetting filtering
    $b.span(:id => "reset_focus").click
    $b.update_icon.wait_while(&:present?)
    assert ( $b.clone_in_list('1').present?), ">> clone 1 is present again"
    assert ( $b.clone_in_list('7').present?), ">> clone 7 is still present"
    assert ( $b.clone_in_list('5').present?), ">> clone 5 is present again"
  end
\end{verbatim}

\question{To unselect them all, you can click in an empty area on the top or
  bottom plot.}
\begin{verbatim}
  def test_B_clones_14_unselect_by_click_in_empty_zone
    # TODO
  end
\end{verbatim}

Sometimes, one wants to hide noisy or unrelated clones.

\question{Select a clone or several clones and click on the \com{hide} button, near the \com{focus} button. Show again these
  clones by clicking on the cross next to the search box.}
\begin{verbatim}
  def test_B_clones_15_hide_selected
    ## test that clone are present before focus
    assert ( $b.clone_in_list('1').present?), ">> clone 1 is present"
    assert ( $b.clone_in_list('7').present?), ">> clone 7 is present"
    assert ( $b.clone_in_list('11').present?), ">> clone 11 is present"
    
    # select somme clones (7, 11, 13)
    $b.clone_in_list("7").click
    $b.clone_in_list("11").click(:control)
    $b.clone_in_list("13").click(:control)
    $b.update_icon.wait_while(&:present?)
    $b.span(:id => "hide_selected").click
    $b.update_icon.wait_while(&:present?)

    ## test that clone are hidden after focus
    assert (     $b.clone_in_list('1').present?), ">> clone 1 is present"
    assert ( not $b.clone_in_list('7').present?), ">> clone 7 is hidden"
    assert ( not $b.clone_in_list('11').present?), ">> clone 3 is hidden"
    assert (     $b.clone_in_list('5').present?), ">> clone 5 is present"

    ## test that clone are present after resetting filtering
    $b.span(:id => "reset_focus").click
    $b.update_icon.wait_while(&:present?)
    assert ( $b.clone_in_list('1').present?), ">> clone 1 is present again"
    assert ( $b.clone_in_list('7').present?), ">> clone 7 is still present"
    assert ( $b.clone_in_list('5').present?), ">> clone 5 is present again"

    ## test that clone are present after resetting filtering
    $b.clear_filter.click
    $b.update_icon.wait_while(&:present?)
    assert ( $b.clone_in_list('1').present?), ">> clone 1 is present again"
    assert ( $b.clone_in_list('7').present?), ">> clone 7 is still present"
    assert ( $b.clone_in_list('5').present?), ">> clone 5 is present again"
  end
\end{verbatim}

% Another way to hide clonesis to assign is to change the tag of it as ``standard (niose)`` and choose to uncheck this tag by clicking on the corresponding tile on the list of tiles at the informatons panel to switch them from a visible state to a filter one.
%%% Voir ci-dessus, déjà mis

It is also possible to filter samples that do not contain a clone.  When you
have lots of samples it helps to keep the sample of interest.  Here the number
of sample is quite limited, so the feature may appear less useful.

\question{\new Click on the \com{IGHV3-11 / IGHJ6} clone in the last sample, whose
  abundance is around 10\,\%.
  Then go in the menu at the upper-right corner of the graph (where \com{5/5}
  is written) and select \com{focus on selected clones}.
}
\begin{verbatim}
  def test_B_clones_16_graphmenu_filter
    assert ( $b.clone_in_list('5').present?), ">> clone 5 is present again"
    div_ratio = $b.span(:id => "visu2_title")
    assert ( div_ratio.text == "5 / 5" ), "Ratio show is correct at init"
    time0 = $b.graph_x_legend("0")
    check0 = $b.checkbox(:id => "visu2_listElem_check_0")
    check2 = $b.checkbox(:id => "visu2_listElem_check_2")
    assert ( time0.present? ), "first sample is present in timeline"
    assert ( check0.set? ), "first checkbox is true"
    assert ( check2.set? ), "third checkbox is true"

    $b.clone_in_list("5").click
    $b.update_icon.wait_while(&:present?) # wait update
    $b.div(:id => 'visu2_menu').hover
    $b.table(:id => "visu2_table").wait_until(&:present?)
    $b.td(:id => 'visu2_listElem_hideNotShare').click
    $b.update_icon.wait_while(&:present?) # wait update

    assert ( div_ratio.text == "3 / 5" ), "Ratio show is correct after focus"
    assert ( not time0.present? ), "first sample is hidden in timeline menu"
    assert ( not check0.set? ), "first checkbox is false"
    assert ( check2.set? ), "third checkbox is true"
  end
\end{verbatim}

By selecting this, the samples where this clone doesn't appear are hidden.
This is useful for instance to assess the contamination among dozens of
samples.

\question{You can go back to the previous view by returning into the menu and
  clicking on \com{show all}. Notice also how in the menu you can select the
  samples to be shown.}
\begin{verbatim}
  def test_B_clones_17_graphmenu_activate
    time0 = $b.graph_x_legend("0")
    check0 = $b.checkbox(:id => "visu2_listElem_check_0")
    check1 = $b.checkbox(:id => "visu2_listElem_check_1")
    check2 = $b.checkbox(:id => "visu2_listElem_check_2")
    div_ratio = $b.span(:id => "visu2_title")
    
    ### starting state
    assert ( div_ratio.text == "3 / 5" ), "Ratio show is correct after focus"
    assert ( not check0.set? ), "first checkbox is false"
    assert ( check2.set? ), "third checkbox is true"

    ### Switch the first sample only (by checkbox)
    $b.div(:id => 'visu2_menu').hover
    $b.table(:id => "visu2_table").wait_until(&:present?)
    check0.click
    $b.until { time0.present? }  # Test "first sample is NOT present in timeline after click"
    assert ( check0.set? ), "first checkbox is true"
    assert ( div_ratio.text == "4 / 5" ), "Ratio show is correct at init"

    ### Click on show all
    $b.td(:id => 'visu2_listElem_showAll').click
    $b.update_icon.wait_while(&:present?) # wait update

    assert ( div_ratio.text == "5 / 5" ), "Ratio show is correct at init"
    assert ( check0.set? ), "first checkbox is true"
    assert ( check1.set? ), "second checkbox is true"
    assert ( check2.set? ), "third checkbox is true"
  end
\end{verbatim}

\section{Analysing clone populations}

\subsection{Clustering clones through inspection of their sequences}

The first thing to be done is to see if some clones should be clustered (because
of sequencing or PCR errors for instance). This step could be automatized
but, in any case, the automatic clustering would need to be checked by an expert
eye.

By default in the bottom plot (the \textit{grid}), the clones
  are displayed according to their V and J genes (or more generally to their
  5' and 3' genes). 

\question{Identify in the grid the clones with an
  \textit{IGHV-3-13}~\textit{IGHJ6} recombination and select them
  all. You can do so either by holding \texttt{Ctrl} or by drawing a rectangle around the clones while
  maintaining down the left button of the mouse.}
\begin{verbatim}
  def test_B_clones_18_select_mulitple_ighv3_13_clonotype
    $b.clone_in_list("0").click
    $b.send_keys 0

    $b.scatterplot_x_legend(8).click # clonotype with IGHV3-13

    ## control that these clonotype is selected and visible in semgenter
    assert ( not $b.clone_in_segmenter('0').present? ), "clone 0 chould not be present";
    assert (     $b.clone_in_segmenter('30').present? ), "clone 30/A should be present";
    assert (     $b.clone_in_segmenter('43').present? ), "clone 43/B should be present";
    assert (     $b.clone_in_segmenter('47').present? ), "clone 43/C should be present";

  end
\end{verbatim}

The sequences of the clones now appear in the bottom part of the browser (the
\textit{sequence panel}). If many clones are selected you can view more sequences
by moving the mouse above the sequence panel.
 In such a case, you may be bothered by the sequence panel going up and
down each time your mouse enters or exits the sequence panel. You can stick it
in its current shape by clicking on the pin at the upper right corner of the
sequence panel.

Then, the sequences in the sequence panel can be visually compared but you can also align
them to see more easily their similarities.


\question{Click on the \com{align} button on the left-hand side. The differences are
emphasized in bold.}
\begin{verbatim}
  def test_B_clones_19_align
    # TODO; align don't available without server
  end
\end{verbatim}

Now it is the user's expertise to determine if sequences are sufficiently
similar, depending on her or his specific question. If some sequences don't appear to be similar enough, you can remove
them from the sequence panel by clicking on the cross in front of the sequence in
the sequence panel.
\question{Remove all the sequences that are not similar enough with the first
  one.}
\begin{verbatim}
  def test_B_clones_20_unselect_from_segmenter
    $b.clone_in_segmenter("47").i(:class => "icon-cancel").click
    $b.update_icon.wait_while(&:present?)
    assert ( not $b.clone_in_segmenter('47').present? ), "clone 47/C should be present";
  end
\end{verbatim}

Now all the sequences in the sequence panel should be highly similar. All their
differences could be due to sequencing or PCR errors.
These artifacts (mutations, homopolymers, insertions, deletions)
depend on the sequencer and the PCR technique.

\question{Cluster all those clones in a single clone by clicking on the ``cluster''
  button, next to the \com{align} button.}
\begin{verbatim}
  def test_B_clones_21_click_on_merge
    $b.merge.click
    $b.update_icon.wait_while(&:present?)

    assert (     $b.clone_in_list("30").present? ),  ">> Real clone A should be present in list "
    assert ( not $b.clone_in_list("43").present? ),  ">> Real clone B should NOT be present in list (clustered)"
    assert (     $b.clone_in_list("47").present? ),  ">> Real clone C should be present in list (not present in segmenter when merged)"
  end
\end{verbatim}

All the clustered sequences now appear within a same clone. That can be seen
in the list: the clone which hosts the subclones appears with a $+$ on its
left. You can click on the $+$ to see the subclones that have been clustered in
the main one.
\question{Click on the $+$ and observe the changes in the grid.}
\begin{verbatim}
  def test_B_clones_22
    # TODO
  end
\end{verbatim}

As you may have noticed the subclones appear again in the grid. You can
compare their sequences again if you'd like (for example to double check that
you were right to cluster them). You can also remove some subclones from the
cluster by clicking on the cross at their left in the list.
\question{For the sake of the exercise, remove the last clone of the cluster.}
\begin{verbatim}
  def test_B_clones_23
    # TODO
  end
\end{verbatim}

\question{%
%For the next step, choose the preset \com{V distribution} (keyboard shortcut \com{5}).
% On n'a pas encore parlé ici des presets. 
Open the \com{cluster} menu, and choose \com{cluster by V/5}. What happened ? There are now two clones with TRGV2. Why ?}
%%  Confirm this by changing the x axis into ``V allele``.
%%% -> Problème, on n'a pas encore parlé des axes à cet endroit.
\begin{verbatim}
  def test_B_clones_24_cluster_by_V
    $b.element(:id => "cluster_menu").hover
    $b.element(:id => "clusterBy_5").wait_until(&:present?)
    $b.element(:id => "clusterBy_5").click

    ## TODO; wait MR !811
  end
\end{verbatim}

\question{In the \com{cluster} menu, select  \com{revert to previous clusters} to undo these clusterings.}
\begin{verbatim}
  def test_B_clones_25
    $b.element(:id => "cluster_menu").hover
    $b.element(:id => "cluster_break_all").wait_until(&:present?)
    $b.element(:id => "cluster_break_all").click
    ## TODO; wait MR !811
  end
\end{verbatim}

\subsection{Other metrics and analysis on the clones}

As a proxy to sequence similarity we used the V and J genes, however there are
other ways to assess sequence similarity that may be more pertinent.
Moreover you may want to plot other metrics on the lymphocyte population.
%
For instance we can choose to plot the V genes versus the length of the N
insertions.
\question{Go to the \com{plot} menu (in the upper left corner of the grid),
  and in the preset box choose \com{V/N length}.}
\begin{verbatim}
  def test_B_clones_26_change_scatter_axis
    # change preset, first time manually as user can do (don't use $b.scatterplot_select_preset this time) 
    select_preset = $b.select_list(:name => "select_preset[]")
    $b.element(:id => "visu_sp_menu").hover
    select_preset.select("V/N length")
    $b.update_icon.wait_while(&:present?)

    assert ( $b.scatterplot_x_label.text == "V/5' gene" ), "Correct legend for axe X for preset V/N length"
    assert ( $b.scatterplot_y_label.text == "N length" ), "Correct legend for axe Y for preset V/N length"
    assert ( $b.scatterplot_x_legend(0).text == "IGHV1-2"), "scatterplot_legend X for this preset is 'IGHV1-2'"
    assert ( $b.scatterplot_y_legend(0).text == "?"), "scatterplot_legend Y for this preset is '0'"
  end
\end{verbatim}

Then you can continue aligning and clustering clones if necessary.

\question{You can also try the preset \com{read length/GC content}
  which tends to separate quite nicely the distinct clones.}
\begin{verbatim}
  def test_B_clones_27_change_preset_2
    $b.scatterplot_select_preset "read length / GC content"
    $b.update_icon.wait_while(&:present?)

    assert ( $b.scatterplot_x_label.text == "Reads length" ), "Correct legend for axe X for preset 0"
    assert ( $b.scatterplot_y_label.text == "GC content" ), "Correct legend for axe Y for preset 0"
    $b.update_icon.wait_while(&:present?)
    assert ( $b.scatterplot_x_legend(0).text == "?"), "scatterplot_legend X pos 0 for this preset is '?'"
  end
\end{verbatim}

Note that you can choose any axis to be plotted: just go the \com{plot} menu and
select any value you would like for the $x$ axis and for the $y$ axis.
For bar charts, the box sizes always relates to the clone size,
and the $y$ axis selects the order of the boxes sharing a same $x$).

%% \item Regarder les stats disponibles, mettre n°7 (taille des reads)

\question{In the \com{plot} menu, switch between the ``bubble plot'' and the ``bar plot''.
In the bar plot mode, pass the mouse over the bars: What happens?}
\begin{verbatim}
  def test_B_clones_28_switch_scatterplot_mode

    skip_on_browser('firefox', '32.0', 'See issue #4595')

    $b.send_keys 0
    $b.update_icon.wait_while(&:present?)
    assert ( $b.scatterplot_x_label.text == "V/5' gene" ), "Correct legend for axe X for preset 0"
    assert ( $b.scatterplot_y_label.text == "J/3' gene" ), "Correct legend for axe Y for preset 0"

    $b.element(:id => "visu_sp_menu").hover
    $b.div(:id => "visu_bar").click
    $b.update_icon.wait_while(&:present?)
    assert ( $b.scatterplot_x_label.text == "V/5' gene" ), "Correct legend for axe X after switch in bar mode"
    assert ( $b.scatterplot_y_label.text == "Size" ), "Correct legend for axe Y after switch in bar mode"

    $b.element(:id => "visu_sp_menu").hover
    $b.div(:id => "visu_plot").click
    $b.update_icon.wait_while(&:present?)
    assert ( $b.scatterplot_x_label.text == "V/5' gene" ), "Correct legend for axe X after switch back in bubble mode"
    assert ( $b.scatterplot_y_label.text == "J/3' gene" ), "Correct legend for axe Y after switch back in bubble mode"
  end
\end{verbatim}

% Another possibility is to request Vidjil to compute the similarity between
% clones.
% \question{Now select the preset \com{plot by similarity} or even \com{plot
%   similarity by locus} to plot similarity for the current locus (beware: this
% may take some time).}
% Now the most similar clones should be close together. However note that it is
% theoretically impossible to achieve such a representation in 2 dimensions. So
% it is possible that two dissimilar clones are close together or, conversely,
% that two similar clones are far apart.

\question{Press the keys \texttt{0} to \texttt{9} on the numeric keypad. What happens ?}
\begin{verbatim}
  def test_B_clones_29_shortcut_preset

    skip_on_browser('firefox', '32.0', 'See issue #4595')

    $b.send_keys 0
    $b.update_icon.wait_while(&:present?)

    assert ( $b.scatterplot_x_label.text == "V/5' gene" ), "Correct legend for axe X for preset 0"
    assert ( $b.scatterplot_y_label.text == "J/3' gene" ), "Correct legend for axe Y for preset 0"
    # assert ( $b.scatterplot_x_legend(0).text == "IGHV1-2"), "scatterplot_legend X at init; IGHV1-2"
    # assert ( $b.scatterplot_y_legend(0).text == "IGHJ1"), "scatterplot_legend Y at init; IGHJ-1"

    $b.send_keys 1
    $b.update_icon.wait_while(&:present?)
    assert ( $b.scatterplot_x_label.text == "V/5' allele" ), "Correct legend for axe X for preset 1"
    assert ( $b.scatterplot_y_label.text == "J/3' allele" ), "Correct legend for axe Y for preset 1"
    #assert ( $b.scatterplot_x_legend(0).text == "IGHV1-2*04"), "scatterplot_legend X with shortcut/preset XXX is: IGHV1-2"
    #assert ( $b.scatterplot_y_legend(0).text == "IGHJ1"), "scatterplot_legend Y with shortcut/preset XXX is: IGHJ-1"


    $b.send_keys 2
    $b.update_icon.wait_while(&:present?)
    assert ( $b.scatterplot_x_label.text == "V/5' gene" ), "Correct legend for axe X for preset 2"
    assert ( $b.scatterplot_y_label.text == "N length" ), "Correct legend for axe Y for preset 2"
    #assert ( $b.scatterplot_x_legend(0).text == "IGHV1-2"), "scatterplot_legend X with shortcut/preset XXX is: IGHV1-2"
    #assert ( $b.scatterplot_y_legend(0).text == "?"), "scatterplot_legend Y with shortcut/preset XXX is: IGHJ-1"

    $b.send_keys 3
    $b.update_icon.wait_while(&:present?)
    assert ( $b.scatterplot_x_label.text == "Reads length" ), "Correct legend for axe X for preset 3"
    assert ( $b.scatterplot_y_label.text == "Locus" ), "Correct legend for axe Y for preset 3"
    #assert ( $b.scatterplot_x_legend(0).text == "20"), "scatterplot_legend X with shortcut/preset XXX is: IGHV1-2"
    #assert ( $b.scatterplot_y_legend(0).text == "IGH"), "scatterplot_legend Y with shortcut/preset XXX is: IGHJ-1"

    $b.send_keys 4
    $b.update_icon.wait_while(&:present?)
    assert ( $b.scatterplot_x_label.text == "Reads length" ), "Correct legend for axe X for preset 4"
    assert ( $b.scatterplot_y_label.text == "Size" ), "Correct legend for axe Y for preset 4"
    #assert ( $b.scatterplot_x_legend(0).text == "20"), "scatterplot_legend X with shortcut/preset XXX is: IGHV1-2"
    #assert ( $b.scatterplot_y_legend(0).text == "20%"), "scatterplot_legend Y with shortcut/preset XXX is: IGHJ-1"

    $b.send_keys 5
    $b.update_icon.wait_while(&:present?)
    assert ( $b.scatterplot_x_label.text == "V/5' gene" ), "Correct legend for axe X for preset 5"
    assert ( $b.scatterplot_y_label.text == "Size" ), "Correct legend for axe Y for preset 5"
    #assert ( $b.scatterplot_x_legend(0).text == "IGHV1-2"), "scatterplot_legend X with shortcut/preset XXX is: IGHV1-2"
    #assert ( $b.scatterplot_y_legend(0).text == "16%"), "scatterplot_legend Y with shortcut/preset XXX is: IGHJ-1"

    $b.send_keys 6
    $b.update_icon.wait_while(&:present?)
    assert ( $b.scatterplot_x_label.text == "N length" ), "Correct legend for axe X for preset 6"
    assert ( $b.scatterplot_y_label.text == "Size" ), "Correct legend for axe Y for preset 6"
    #assert ( $b.scatterplot_x_legend(0).text == "?"), "scatterplot_legend X with shortcut/preset XXX is: IGHV1-2"
    #assert ( $b.scatterplot_y_legend(0).text == "55%"), "scatterplot_legend Y with shortcut/preset XXX is: IGHJ-1"

    $b.send_keys 7
    $b.update_icon.wait_while(&:present?)
    assert ( $b.scatterplot_x_label.text == "CDR3 length" ), "Correct legend for axe X for preset 7"
    assert ( $b.scatterplot_y_label.text == "Size" ), "Correct legend for axe Y for preset 7"
    #assert ( $b.scatterplot_x_legend(0).text == ""), "scatterplot_legend X with shortcut/preset XXX is: IGHV1-2"
    #assert ( $b.scatterplot_y_legend(0).text == ""), "scatterplot_legend Y with shortcut/preset XXX is: IGHJ-1"

    $b.send_keys 8
    $b.update_icon.wait_while(&:present?)
    assert ( $b.scatterplot_x_label.text == "J/3' gene" ), "Correct legend for axe X for preset 8"
    assert ( $b.scatterplot_y_label.text == "Size" ), "Correct legend for axe Y for preset 8"
    #assert ( $b.scatterplot_x_legend(0).text == ""), "scatterplot_legend X with shortcut/preset XXX is: IGHV1-2"
    #assert ( $b.scatterplot_y_legend(0).text == ""), "scatterplot_legend Y with shortcut/preset XXX is: IGHJ-1"

    $b.send_keys 9
    $b.update_icon.wait_while(&:present?)
    assert ( $b.scatterplot_x_label.text == "Size" ), "Correct legend for axe X for preset 9"
    assert ( $b.scatterplot_y_label.text == "Size (other)" ), "Correct legend for axe Y for preset 9"
    #assert ( $b.scatterplot_x_legend(0).text == ""), "scatterplot_legend X with shortcut/preset XXX is: IGHV1-2"
    #assert ( $b.scatterplot_y_legend(0).text == ""), "scatterplot_legend Y with shortcut/preset XXX is: IGHJ-1"

    $b.send_keys 0
    $b.update_icon.wait_while(&:present?)
    assert ( $b.scatterplot_x_label.text == "V/5' gene" ), "Correct legend for axe X for preset 0"
    assert ( $b.scatterplot_y_label.text == "J/3' gene" ), "Correct legend for axe Y for preset 0"
    #assert ( $b.scatterplot_x_legend(0).text == "IGHV1-2"), "scatterplot_legend X with shortcut/preset XXX is: IGHV1-2"
    #assert ( $b.scatterplot_y_legend(0).text == "?"), "scatterplot_legend Y with shortcut/preset XXX is: IGHJ-1"
  end
\end{verbatim}

There is still a feature to help you analyse your data that we have not
explored yet.
You can change the colors to make it represent some variables of interest
with the \com{color by} menu.
\question{First choose the preset \com{V/J (genes)} and
  then color by \com{N length} (in the box at the top of the screen).}
\begin{verbatim}
  def test_B_clones_30_color_mode
    $b.scatterplot_select_preset "V/J (genes)"
    $b.update_icon.wait_while(&:present?)

    # control color state at init
    assert ( not $b.info_colorBy.span(:class => "gradient").exist? ), "at init, color are not gradient"

    select_color = $b.select_list(:id => "color_menu_select")
    select_color.select("N length")
    $b.update_icon.wait_while(&:present?)
    
    ## control
    $b.clone_in_list("0").click
    assert ($b.clone_info('1')[:name].style('color') ==  'rgba(0, 189, 225, 1)' ), "Good color for clone 1 with color method N"
    assert ($b.clone_info('4')[:name].style('color') ==  'rgba(0, 173, 251, 1)' ), "Good color for clone 4 with color method N"
    ## test label
    assert ( $b.scatterplot_x_legend(0).text == "IGHV1-2"),  "scatter plot legend X is correct after preset change"
    assert ( $b.scatterplot_y_legend(0).text == "IGHJ1"), "scatter plot legend Y is correct after preset change"
  end
\end{verbatim}
  
\marginpar{We apologize to color blinds: the colors are not yet color-blind friendly.}Clones that are close on the grid with similar colors are likely to
be similar.

\question{Choose now the preset \com{CDR3 length distribution} and
  then color by \com{size}.
  See that the color tiles in the info part (upper right) change to show the color key.}
\begin{verbatim}
  def test_B_clones_31_color_gradient
    $b.scatterplot_select_preset "CDR3 length distribution"
    select_color = $b.select_list(:id => "color_menu_select")
    select_color.select("Size")
    $b.update_icon.wait_while(&:present?)

    assert ( $b.info_colorBy.span(:class => "gradient").exist? ), "color gradient is present"
  end
\end{verbatim}

\question{ Instead of coloring by clone size, you could also color by
  \com{clonotype}. When coloring by \com{clonotype}, each clone has a random color. Thus in
  a bar plot, it is a convenient color mode to see the peaks that are due to a
  single clone or to several clones.
  However clones may be very similar. Another option is to color by
  \com{CDR3}. In such case all clones with the same CDR3 will have the same
  color (note that, due to a lack of available colors two different CDR3s
  could share the same color just by chance).}
\begin{verbatim}
  def test_B_clones_32_color_by_clone
    select_color = $b.select_list(:id => "color_menu_select")
    select_color.select("Clonotype")
    $b.update_icon.wait_while(&:present?)

    $b.clone_in_list("0").click
    assert ($b.clone_info('1')[:name].style('color') ==  'rgba(153, 153, 61, 1)' ), "Good color for clone 1 with color method 'clone'"
    assert ($b.clone_info('4')[:name].style('color') ==  'rgba(61, 61, 153, 1)' ), "Good color for clone 4 with color method 'clone'"
  end
\end{verbatim}


Using those different features you should be able to analyse how similar your
sequences are, and potentially you could cluster them if you'd like or tag them.

\question{
  Select the most abundant clone. It now appears in the sequence panel.
  Now we would like to compare the sequence with the germline genes.
  We can add the germline genes to the sequence panel by going
  to the \com{import/export} menu and by clicking on \com{add germline genes}.
  Now we can click on the \com{align} button to see the alignment between the
  genes and the sequence. Mutations can be identified and silent mutations are
  displayed with a double border in blue.
}
\begin{verbatim}
  def test_B_clones_33_align_germline
    assert ( not $b.clone_in_segmenter("IGHJ6*02").exist? ), "sequence germline is NOT present in the segmenter"
    $b.clone_in_list("1").click

    $b.menu_import_export.hover
    $b.menu_item_export_add_germline.click
    $b.update_icon.wait_while(&:present?)

    assert ( $b.clone_in_segmenter("IGHJ6*02").exist? ), "sequence germline is present in the segmenter"

    # todo, align; mutation; not available without server side
  end
\end{verbatim}

\bigskip

\textit{This part is specific to samples analyzed with Vidjil-algo.}

Some clones may be less trustable than other ones\dots{} Let's see how to spot them.
\question{In the clone list, search clones with an orange warning at the
  right side. Click on the warning. What are the warnings due to?}
\begin{verbatim}
  def test_B_clones_34
    # click outside of menu_item_export to close it
    $b.graph_x_legend("0").click

    $b.clone_in_list("11").click #first clone with warning

    # <i title="W53: Similar to clone #2 - TRGV10*01 13//5 TRGJP1*01undefined: undefinedundefined: undefinedundefined: undefinedundefined: undefined" class="icon-warning-1"></i>

    assert ( not $b.clone_info("1")[:info].i(:class => "icon-warning-1").exist?), "no warning icon for clone 1"
    assert ( $b.clone_info("11")[:info].i(:class => "icon-warning-1").exist?), "warning icon present for clone 11"
    assert ( not $b.modal_container.div(:id => "info_window").present? ), "modal info of clone is not present"
    $b.clone_info("11")[:info].click

    modal_table = $b.modal_container.div(:id => "info_window")
    
    assert ( $b.modal_container.div(:id => "info_window").present? ), "modal info of clone is present"
    assert ( modal_table.td(:text => "W53").present? ), "warning number present in the clone info table"
    assert ( modal_table.td(:text => "Similar to clone #2 - TRGV10*01 13//5 TRGJP1*01").present? ), "warning content present in the clone info table"

    # close modal
    $b.modal_container.span(:class => 'closeButton').click

  end
\end{verbatim}

There may have several reasons: 
\begin{itemize}
\item average coverage: in that case the clonal sequence displayed is short
  compared to the reads in the clone. This may be the case when too different
  sequences have been put in a clone. The value is generally $\geq 80\,\%$.
\item $e$-value: It is a statistical value computed to ensure that
  recombinations have not been spot by chance. This value is generally much
  lower than 1 ($<10^{-5}$).
\item Clone similar to another one: In that case Vidjil-algo tells you that
  other clones have the same genes and may be similar
\item Non-recombined sequences: Some known unrecombined sequences are tagged
  so that you can spot them easily. We tag the unrecombined IGHD7-27/IGHJ1
  sequences that may be amplified.
\end{itemize}

You can view those values for any clone by clicking the \textit{i} icon on the
right side, in the list of clones.
\subsection{Analysing recombinations from several loci}

First make sure to come back to the preset \com{V/J (genes)} in the \com{plot} menu.

If you want to focus on specific locus, you can click on the locus name in
the upper left part. One click will make the locus disappear, another one will
make it appear again.
If you hold the \texttt{Shift} key (the one which is usually above the left
\texttt{Ctrl} key) while clicking it will hide all the loci but the one you
clicked on.

\question{Click on \com{IGH}, while holding the \texttt{Shift} key. Now what is the
  number of analyzed reads? Why did it change?}
\begin{verbatim}
  def test_B_clones_35_locus_switch
    assert ( not $b.locus_topleft("TRG").classes.include? "unchecked" ), "locus TRG present in info panel"
    assert ( not $b.locus_topleft("IGH").classes.include? "unchecked" ), "locus IGH present in info panel"
    $b.locus_topleft("IGH").click(:shift)
    $b.update_icon.wait_while(&:present?)

    assert ( $b.locus_topleft("TRG").classes.include? "unchecked" ), "locus TRG NOT present in info panel after switch"
    assert ( not $b.locus_topleft("IGH").classes.include? "unchecked" ), "locus IGH present in info panel after switch"
  end
\end{verbatim}

\question{Now click on \com{TRG}, to filter it in again.}
\begin{verbatim}
  def test_B_clones_36_activate_again_locus
    $b.locus_topleft("TRG").click
    $b.update_icon.wait_while(&:present?)
    assert ( not $b.locus_topleft("TRG").classes.include? "unchecked" ), "locus TRG present in info panel after switch"
    assert ( not $b.locus_topleft("IGH").classes.include? "unchecked" ), "locus IGH present in info panel after switch"
  end
\end{verbatim}

\question{Press on the \texttt{g} key. What happens? Now, press on the
  \texttt{h} key. Press on the \texttt{g} again (you can do that anytime you
  like :)). Let's stick to the TRG locus.}
\begin{verbatim}
  def test_B_clones_37_shortcut
    $b.send_keys 0 # return to a grap with locus 
    $b.update_icon.wait_while(&:present?)
    assert ( $b.scatterplot_x_legend(0).text == "IGHV1-2"), "scatterplot_legend X at init; IGHV1-2"
    assert ( $b.scatterplot_y_legend(0).text == "IGHJ1"), "scatterplot_legend Y at init; IGHJ-1"

    $b.send_keys "g"
    $b.update_icon.wait_while(&:present?)
    assert ( $b.scatterplot_x_legend(0).text == "TRGV2"), "scatterplot_legend X  after shortcut G; TRGV2"
    assert ( $b.scatterplot_y_legend(0).text == "TRGJ1"), "scatterplot_legend Y  after shortcut G; TRGJ1"
    

    $b.send_keys "h"
    $b.update_icon.wait_while(&:present?)
    assert ( $b.scatterplot_x_legend(0).text == "IGHV1-2"), "scatterplot_legend X after shortcut H; IGHV1-2"
    assert ( $b.scatterplot_y_legend(0).text == "IGHJ1"), "scatterplot_legend Y after shortcut H; IGHJ-1"
  end
\end{verbatim}

You can also change the current locus by clicking on the locus name in the
right part of the grid.
\begin{verbatim}
  def test_B_clones_38_switch_locus_by_scatterplot_click
    $b.send_keys "h"
    $b.update_icon.wait_while(&:present?)
    
    $b.send_keys 0 # return to a grap with locus 
    $b.update_icon.wait_while(&:present?)
    assert ( $b.scatterplot_x_legend(0).text == "IGHV1-2"), "scatterplot_legend X at init; IGHV1-2"
    assert ( $b.scatterplot_y_legend(0).text == "IGHJ1"), "scatterplot_legend Y at init; IGHJ-1"

    $b.scatterplot_locus("TRG").click
    $b.update_icon.wait_while(&:present?)
    assert ( $b.scatterplot_x_legend(0).text == "TRGV2"), "scatterplot_legend X  after shortcut G; TRGV2"
    assert ( $b.scatterplot_y_legend(0).text == "TRGJ1"), "scatterplot_legend Y  after shortcut G; TRGJ1"  
  end
\end{verbatim}

\subsection{Clone quantification (using spike-ins)}

Sometimes you may include spike-ins in your sample to allow a more reliable
quantification.
Let us assume that the main clone with IGHV-3-9 / IGHJ5 is a spike-in whose
expected concentration is 1\% (.01).

\question{First let's color this clone with the \com{standard} tag.}
\begin{verbatim}
  def test_B_clones_39_reset_color_tag
    select_color = $b.select_list(:id => "color_menu_select")
    select_color.select("Tag")

    # New color
    cinfo = $b.clone_info("454")
    assert ($b.clone_info('0')[:name].style('color') ==  'rgba(108, 113, 196, 1)' ), "color: "+$b.clone_info('0')[:name].style('color')
  end
\end{verbatim}

\question{ Now we will set its concentration to .01 as expected. Click again on
  the star. In the \com{normalize to} field enter \com{.01} and click \com{ok}.
  Now, in the graph, this clone should correspond to a straight line at 1\%.}
\begin{verbatim}
  def test_B_clones_40_normalization
    cinfo = $b.clone_info("1")
    assert ( cinfo[:size].text == "9.665%"), "correct size after normalization"
    ## open tag menu for clone 1
    $b.clone_info("1")[:star].click # top IGH
    ## change norm value
    $b.tag_selector_edit_normalisation.send_keys "0.01"
    $b.tag_selector_normalisation_validator.click
    # $b.send_keys :enter
    $b.update_icon.wait_while(&:present?)

    ## control
    assert ( cinfo[:size].text == "1.000%"), "correct size after normalization"
    $b.menu_settings.hover
    assert ($b.div(:id => 'normalizetest1').present?), "Form have the input for expected normalization"
  end
\end{verbatim}

\question{ Notice how the concentrations of the other clones have changed
  accordingly.
  You can go to the \com{settings} menu to disable this normalization and to
  go back to the raw concentrations.}
\begin{verbatim}
  def test_B_clones_41_disable_normalization
    cinfo = $b.clone_info("1")
    assert ( cinfo[:size].text == "1.000%"), "correct size after normalization (init of test)"
    
    $b.menu_settings.click
    $b.input(:id => 'reset_norm').click

    ## control
    cinfo = $b.clone_info("1")
    assert ( cinfo[:size].text == "9.665%"), "correct size after disabled normalization"
  end
\end{verbatim}

Then you can set expected concentrations for other clones and you are free to
switch between those normalizations.
It is also possible to set up normalization against external data,
contact us if you are interested.

